\chapter{AGREEMENT ATTRACTION} \label{ch:accounts}

The errors in the subject-verb dependency described in Chapter \ref{ch:intro} have been previously noted by grammarians (Quirk, Greenbaum, Leec, \& Svartvik, 1972). However, accounts that try to explain the mechanism behind these errors were not introduced before the pioneering experimental work conducted by \citet{BockMiller:1991}. This chapter presents accounts of agreement attraction and influential studies that led to the formation of these accounts. Due to the scope of the thesis, we do not report all the experimental work conducted in number agreement attraction. We also do not report studies focusing on gender or case attraction. The studies we introduce in this chapter are the ones that provided some generalizations within the number agreement attraction field and contributed to the formation of new accounts.



\section{Feature percolation account} \label{sec:feature_perc}

The first account that tried to explain agreement attraction effects was the Feature Percolation account \citep{BockEberhard93}. Bock and her colleagues conducted many studies that led to the formation of this account. Many of these studies had a focus on sentence production. For example, the first study conducted was \cites{BockMiller:1991} study. They ran three production studies using a sentence completion task. After hearing the preamble, participants were asked to complete the sentence. In their first experiment, they manipulated the length of the preamble (short x long), the number-marking of the attractor (plural x singular) and the head noun (plural x singular), and the type of the attractor (object relative clause x subject relative clause x prepositional phrase (to) x prepositional phrase (on)). As a result, they had 16 conditions. One set of example sentences is provided in (\ref{ex:bm_exp1short}) and (\ref{ex:bm_exp1long}).

\ea {Short Preambles} \label{ex:bm_exp1short}
  \ea {Object Relative Clause} \\* The {key(s)} to the {cabinet(s)} \ldots
  \ex {Subject Relative Clause} \\* The {boy(s)} that liked the {snake(s)} \ldots
  \ex {Prepositional Phrase (to)} \\* The {soldier(s)} that the {officer(s)} accused \ldots
  \z
\ex {Long Preambles} \label{ex:bm_exp1long}
  \ea {Object Relative Clause} \\* The {key(s)} to the ornate Victorian {cabinet(s)} \ldots
  \ex {Subject Relative Clause} \\* The {boy(s)} that liked the colorful garter {snake(s)} \ldots
  \ex {Prepositional Phrase (to)} \\* The {soldier(s)} that the battalion's senior {officer(s)} accused \ldots
  \z
\z

\citeand{BockMiller:1991} found that participants mainly made agreement errors and completed the preamble with an erroneously marked verb when the head noun is singular and the attractor is plural. The errors were negligible when the head noun is singular and when the attractor and the head noun matched in number. They also find that participants made more errors when the attractor was in a prepositional phrase rather than a relative clause. They did not find a substantial difference between the short and long preambles and between prepositions type or relative clauses types. Other two experiments tested the effect of animacy and found that animacy did not amplify the attraction only when the agreement controller was easily distinguished. They found that when there are more than one subject as in relative clause conditions, animate ones are erroneously designated as an agreement controller and induced attraction. This was not the case with prepositional constructions. They also tested the direction of attraction. In Experiment 3, they used nominal heads modified with relative clauses. They only provided the subject of relative clauses as in `\emph{The colonies that the king \ldots{\ }},' and asked participants to complete the sentence. They found that while participants did not make agreement errors in determining the marking on the embedded verb, they made errors on the matrix verb with sentence fragments like  `\emph{The colony that the kings \ldots{\ }}.' Thus, they inferred that syntactically higher elements could not impact the number information of the syntactically more embedded element, but the other way around was possible.

\citeand{BockCutting1992} have tested whether the syntactic position of the attractor played a role. They have conducted three production studies with sentence completion task and showed that both the complexity of the phrase the attractor resides in, and its relation to the head noun affected the attraction errors. Even though both complement clauses such as `\emph{The report that they controlled the fires \ldots{\ }}' and the relative clauses such as `\emph{The editor who rejected the books \ldots{\ }}' triggered more erroneous agreement on the verb compared to their singular attractor counterparts, neither of these constructions disrupted the agreement process as prepositional phrases such as `\emph{The editor of history books \ldots{\ }}' did.

Later, \citeand{BockEberhard93} conducted several experiments to test the effects of notionally plural nouns such as \emph{fleet-ship} and pseudoplurals whose endings match with the plural marking in English such as \emph{cruise}, and irregular plurals such as \emph{mice-mouse}. These experiments were again production experiments with a sentence completion task. They have found that neither pseudoplurals nor notionally plural collective nouns as attractors lead participants to make agreement errors. The error rate in pseudoplurals and collective nouns was comparable to the nouns with no phonological resemblance to English plural endings and non-collective nouns. On the other hand, they have found that irregular plural marking resulted in similar percentages of agreement errors to regular plural marking in the conditions with singular heads and plural attractors.

Along with these production studies, \citeand{NicolEtAl1997} attested similar agreement attraction effects in a comprehension study. They conducted a maze\footnote{A maze task is an experimental method in which participants are read the stimuli in a word-by-word fashion similar to self-paced reading or speeded acceptability judgment. In contrast to these methods, participants are prompted with two words at each reading instance and asked to choose the correct word.} and speeded grammaticality judgment task using sentences like those in (\ref{ex:nicol97}). They have manipulated the number-marking of the attractor (plural x singular) and the head noun (plural x singular). They also manipulated the number-marking of the verb (plural x singular); however, the ungrammatical items were not included in the experiment. 

\ea \label{ex:nicol97}
  \ea {Singular Head \& Singular Attractor} \\* The {author} of the {speech} is here now.
  \ex {Singular Head \& Plural Attractor} \\* The {author} of the {speeches} is here now.
  \ex {Plural Head \& Singular Attractor} \\* The {authors} of the {speech} are here now.
  \ex {Plural Head \& Plural Attractor} \\* The {authors} of the {speeches} are here now.
  \z
\z

In their first experiment, where they used a maze task, they measured reaction times and found that participants had more difficulty and spent more time when the number marking on the attractor and the head noun mismatched. However, this effect was only present when the head noun was singular. In their second experiment, a comprehension task, they have used the same manipulations as the maze task. They again did not include ungrammatical items. The results of the comprehension task verified their findings in the maze task: participants had processing difficulty only in the conditions where the head is singular and the attractor is plural. 

\citeand{PearlmutterGarnseyBock:1999} conducted another three experiments using self-paced readings ac'nd eye-tracking, where they found comparable results to previous attraction findings. They have manipulated the number of the attractor (plural x singular) and the verb (plural x singular) in their items. They kept the number of the head constant: it was always singular. The conditions they used later became the mainstream conditions in agreement attraction experiments. One set of conditions can be found in (\ref{ex:pearl99}).

\ea \label{ex:pearl99}
  \ea[*]{{Plural Attractor \& Ungrammatical (Plural Verb)} \\* The {key} to the {cabinets} were rusty from many years of disuse.}
  \ex[]{{Plural Attractor \& Grammatical (Singular Verb)} \\* The {key} to the {cabinets} was rusty from many years of disuse.}
  \ex[*]{{Singular Attractor \& Ungrammatical (Plural Verb)} \\* The {key} to the {cabinet} were rusty from many years of disuse.}
  \ex[]{{Singular Attractor \& Grammatical (Singular Verb)} \\* The {key} to the {cabinets} was rusty from many years of disuse.}
  \z
\z

Their results in self-paced reading experiment showed a main effect of attractor number on readings times of the regions immediately following the verb, that is \emph{rusty}. Their results showed that the plural marking on the attractor increased the readings times in grammatical sentences amd reduced the reading times in ungrammatical sentences. The presence of a plural attractor made participants process the ungrammatical (plural verb) sentences faster and slowed the processing in grammatical (singular verb) sentences.  They have verified their findings with eye-tracking experiments, which showed the main effect of attractor number on regressive saccades, first-pass residual reading times, and total reading times. Interestingly, in all their experiments, the presence of a plural attractor increased the reading time in grammatical sentences but reduced RTs in ungrammatical sentences. 

Together, these findings raised certain generalizations regarding the agreement attraction phenomenon. 

\ea {Generalizations:} \label{perc_generalizations}
  {\doublespacing \begin{enumerate}[label=\roman*.]
    \item Noun semantics did not make any difference in the proportion of errors. While singular collective nouns as attractors did not trigger any attraction effects, plural animate nouns as attractors did not create additional effects compared to plural inanimate nouns.
    \item Nouns with morpho-phonological similarities to plural endings were not effective attractors. Pseudoplurals created comparable attraction errors to usual singular items that do not end with one of the possible plural allomorphies in English.  
    \item While the hierarchically lower element can influence the representation of the hierarchically higher element, the other way around is not possible. The features cannot percolate down, but can percolate upwards.
  \end{enumerate}}
\z


In the light of these studies and generalizations, Bock and her colleagues proposed the Feature Percolation account of agreement attraction \citep{BockMiller:1991,BockCutting1992,BockEberhard93}. The main workhorse of this account is the feature copying/migration mechanism. Since they found that collective nouns were not competent attractors, they argued that agreement attraction operated with grammatical features that are only interpretable by syntax. Also, the non-existent effect of phonological manipulations strengthens the idea that agreement attraction was a syntax-only and phonology-free phenomenon. The last generalization from (\ref{perc_generalizations}) suggested that the number feature can only move upwards in the syntax tree. Lastly, the first experiments where the plural head and singular attractor combinations were used showed that while the presence of a plural attractor in mismatch conditions (where the attractors and heads number mismatches) can affect the error rates, the presence of a singular attractor in mismatch conditions does not result in agreement errors. This discrepancy is interpreted as an evidence towards the markedness of the plurality. While our parser/syntax specifies the plurality in the feature set, being singular is represented as a lack of a feature.

Having settled the findings and an account that can cover these findings, we can spell out the step-by-step generation of the agreement attraction phenomenon according to the Feature Percolation account. We will take the phrase `\emph{The key to the cabinet \ldots{\ }}' first. The singular head and singular attractor (SS configuration) does not have plural in their feature combinations, and they should have a matching set of features in terms of number. Thus, no percolation should occur. Every agreement error found in this baseline condition should be due to attentional lapse. 

The PP configuration is also similar to the SS configuration. Since both nouns have matching features, there will be no percolation of features. Additionally, this account has a binary understanding of plurality; we cannot treat the plurality as a continuum. Thus, there cannot be more plural items than the plurals. Thus, having two plural features within the same phrase will not affect the attraction phenomenon.

When we have a PS configuration as in `\emph{The keys to the cabinet \ldots}', the Feature Percolation account suggests that while the head noun \emph{keys} has a plural feature, the attractor noun \emph{cabinet} neither has a plural nor a singular feature. This is due to the markedness effect, only the more marked features are marked in this uniary system. Therefore, we do not have anything that can percolate to the head noun. Moreover, the feature of the head noun \emph{keys} cannot not percolate down to the attractor. Thus, according to the Feature Percolation account, there should be no additional error in this configuration when we compare it to the baseline SS condition.

However, when we have an SP configuration as in `\emph{The key to the cabinets \ldots{\ }}', an increased proportion of agreement errors is expected compared to the other configurations. The main reason for this increase is that the feature plural may percolate upwards or be copied to the feature set of the head noun \emph{key} on some occasions. After this percolation, the whole subject phrase's grammatical number is changed to plural from the initial singular state. Since attraction occur at the level of syntax in the Feature Percolation account, and we operate over binary features, it is expected that the newly formed plural subject will act as an agreement controller instead of the initial form. In this account, the reason for agreement attraction is the malformed representation of the complex DP. 

If we consider the comprehension side of this story, we again expect fewer errors in acceptability judgments when the subject head and the attractor have a matching number marking as in `\emph{The key to the cabinet is \ldots{\ }}' and `\emph{* The key to the cabinet are \ldots{\ }}'. The critical thing to note about comprehension is that the plurality on the head noun is not manipulated and the head noun is typically left singular following \citeand{PearlmutterGarnseyBock:1999}. Studies mostly compare mismatched conditions (Singular head, Plural attractor) in ungrammatical and grammatical sentences to the matched conditions.

In the comprehension of mismatching conditions in ungrammatical sentences like `\emph{* The key to the cabinets are \ldots{\ }}', we expect an increased percentage of erroneous judgments compared to a matching condition (SS) in ungrammatical sentences following the Feature Percolation account. This is due to the hypothesized copying of the feature plural to the subject head or the root node of the complex DP. When the feature is percolated upwards on some occasions, the mismatch between the subject head and the verb will not create any disturbance in the processing of the sentence. Since this percolation is not dependent on any participant or item, we expect to see these errors with most of the participants systematically. Moreover, these errors should not be born out of trial order or any particular semantics of any sentence. 

The exact process is expected in grammatical sentences with mismatching conditions. Since agreement attraction is due to the malformed representations of the subject phrase in the Feature Percolation account, the number marking on the verb should not matter. When we have a plural attractor \& singular head configuration (SP) with a singular verb, as in `\emph{The key to the cabinets is \ldots{\ }}', the plural feature of the attractor should be copied to the head noun on some occasions as well. Thus, while we expect to see more yes responses in ungrammatical sentences with mismatching conditions, we should see fewer yes responses in grammatical sentences with mismatching conditions than their matching condition counterparts. 


\section{Marking \& morphing account} \label{sec:mm}

After the initial findings that led to the Feature Percolation account, many researchers have tried to replicate these findings with different constructions in various languages. While some of the generalizations held against these additional experimental works, most of them were challenged, and agreement attraction was found to be more nuanced than the initial picture. 

For instance, one of the basic assumptions of the Feature Percolation theory was that the percolation occurs upwards, within the same phrase, and between nouns. \citeand{HatsuikerEtAl2001} tested whether agreement attraction is restricted to these syntactic specifications. They have conducted three production experiments using sentence-completion tasks and tested whether plural nominal direct objects and direct-object pronouns culminate in attraction effects. They provided preambles like the ones in (\ref{ex:hartsuiker1}). They manipulated the attractor number (plural x singular) and the syntactic function of the attractor (subject-modifying x direct-object). The attractor is provided within a prepositional phrase in the subject-modifier condition, similar to previous agreement attraction studies. 

\ea \label{ex:hartsuiker1}
  \ea \label{ex:hartsuiker1_subj_mod} {Subject-Modifier condition}\\*
    \gll Karin zegt dat het {meisje} met de {krans(-en)} \ldots \\
    Karin says that the girl with the garland(-\Pl{}) \ldots \\
    \glt `Karin says that the girl with the garland/garlands \ldots'
  \ex \label{ex:hartsuiker1_dobj} {Direct Object condition}\\*
    \gll Karin zegt dat het meisje de krans(-en) \ldots \\
    Karin says that the girl the garland(-\Pl{}) \ldots \\
    \glt `Karin says that the girl VERB the garland/garlands.'
  \z
\z

They found that participants produced verbs with wrong number marking more often when the attractor is plural. This effect was observed both in subject-modifier and direct-object conditions. However, the magnitude of the effect was more considerable in subject-modifier conditions. These results showed that the agreement controller and the attractor did not need to share a dominating node: direct objects could also interfere with the subject-verb dependency. The Feature Percolation account, which comes with a strong hypothesis of attraction being limited to the subject phrase, would predict no attraction effect since the feature plural of the attractor cannot percolate to the subject from the direct object position. 

Additionally, different syntactic functions also influenced agreement attraction. The difference between the way from the PP-modifier to the subject head and the object and to the subject head through the syntax tree matters in attraction. One way to formalize this difference is to put it in the form of `syntactic distance.' One may think of syntactic distance in many different ways. The number of nodes, the number of phrases, or the number of spans can be used for calculating syntactic distance. If we take nodes as a measuring unit, we can say that a DP within a PP-modifier of the subject is syntactically closer to the a DP functioning as a direct object.

In addition to \cites{BockCutting1992} work, Franck, Vigliocco, and Nicol (2002) conducted two experiments to test the effect of syntactic distance on agreement attraction in French and English. They conducted production experiments with a sentence-completion task using sentence preambles as in (\ref{ex:franck02}). They have used three DPs in the preamble, where the first one (DP$_1$) is the agreement controller, and two other DPs (DP$_2$ and DP$_3$) are embedded in prepositional phrases. They have manipulated the number marking on all DPs in their experiment. 

\ea \label{ex:franck02} The {threat/threats} to the {president/presidents} of the {company/companies} \ldots \z

The important detail of their experimental item was that the PP that contains the third DP (DP$_3$) modifies DP$_2$ while the PP with DP$_2$ modifies modifies DP$_1$. Tree in (\ref{tree:franck02}) shows the recursive embedding in a simplified fashion. By embedding the possible attractors deeper, they aimed to check whether the syntactic or the local distance is more effective. If attraction effects were more prevalent in the conditions where only the local noun (DP$_3$) is plural (SSP configuration) compared to the ones where only the syntactically closer DP (DP$_2$) is plural (SPS configuration), it would support the idea that linear proximity to the verb is more important than the syntactic proximity to the head subject.

\ea \label{tree:franck02} {Recursive PP Embedding}\\*
\begin{forest}
  [DP
    [D\\the]
    [NP
      [NP
        [N\\threat(s)]
      ]
      [PP
        [P\\to]
        [DP
          [D\\the]
          [NP
            [NP
              [N\\president(s)]
            ]
            [PP
              [P\\of]
              [DP
                [D\\the]
                [NP [N\\company(s)]]
              ]
            ]
          ]
        ]
      ]
    ]
  ]
\end{forest}
\z

They found that participants made more agreement errors in SPS configurations than in SSP configurations. Participants did very few errors in SSP configurations. In the configurations where the controller is plural and the only noun with a mismatching number marking is DP$_2$ (PSP), participants again made more agreement errors than in PPS configurations. The results were comparable in the English and French experiments. Their findings were incompatible with the previous explanations of agreement attraction. One previous explanation was that an intervening noun induced the attraction effects for locality reasons (Fayol, Largy, \& Lemaire, 1994; Quirk et al., 1972). The locality view predicted more agreement errors in PPS or SSP configurations than in PSP or SPS configurations. Another previous explanation suggested that all DPs within the subject phrase were equally possible to interfere with the subject-verb agreement \citep{BockCutting1992}. According to this view, both interfering DPs should have a comparable impact on the agreement error percentages, which was not the case. From their finding, it was clear that the syntactic relations between the head and the controller are crucial aspects of the agreement attraction phenomenon. \citeand{FranckEtAl2002} argues that their results support the idea that attraction occurs at a point when the features are ordered hierarchically.

Another tenet of the Feature Percolation account was the difference between the effects of notional and grammatical numbers. The previous findings showed that collective nouns or distributivity did not trigger attraction effects. However, additional experiments conducted in Dutch \citep{ViglioccoEtAl96b}, French \citep{ViglioccoEtAl96b}, and English \citep{HumphreysBock2005,HaskellMacDonald2003,Eberhard1999} presented conflicting results with the previous \cites{BockMiller:1991} findings. It was found that when the sentence is accompanied by a visual representation of the initial DPs, the distributivity gave rise to higher agreement errors \citep{ViglioccoEtAl96b}. Moreover, the syntactic role of the collective pronoun influenced the attraction effects. While collective nouns as attractors did not interfere with the subject-verb dependency, singular collective nouns as agreement controllers amplified the agreement errors \citep{HaskellMacDonald2003}.

\citeand{ViglioccoEtAl96b} conducted two production experiments on Dutch and French. They used sentence-completion tasks like previous production experiments. However, they also presented the preambles as a picture. They manipulated the number of the attractor (plural x singular) and the presentation of the preambles (single-token x multiple-token). One set of experimental items used in their experiment is presented in (\ref{ex:vigli96b}). In single-token conditions, there will be only one strike to either one or multiple ministers depending on the number marking of the attractor. In multiple-token conditions with singular attractors, there will be a single picture on each mug. 

On the other hand, when the attractor is plural, the presentation included multiple mugs with a picture on them. They choose the attractors so that it is semantically implausible to imagine a non-distributive reading in multiple-token conditions with mismatching number markings. For example, it is very odd to think there is a single picture stretched over multiple mugs. 

\ea \label{ex:vigli96b}
  \ea {Single Token} \label{ex:vigli96b-single}\\*
    \gll De {aanslag} op de {minister(-s)} \ldots{}\\
    the strike on the minister(-\Pl{}) \ldots{}\\
    \glt `The strike on the minister \ldots{}'
  \ex {Multiple Token} \label{ex:vigli96b-multiple}\\*
    \gll De {afbeelding} op de {mok(-ken)} \ldots{} \\
    the picture on the mug(-\Pl{}) \ldots{} \\
    \glt `The picture on the mog \ldots{}'
  \z
\z


They found that agreement errors were more common in multiple-token conditions with mismatching number marking (SP configuration). Even though there was an effect of a plural attractor in single-token conditions, it was smaller than the one with multiple-token conditions. The same effect of multiple-token conditions was also observed in the French experiment. These findings contradict with the predictions of the Feature Percolation account, which claims that only the grammatical number is relevant to attraction effects.

In addition to distributivity effects, \citeand{HaskellMacDonald2003} tested how collective nouns that are notionally plural impact agreement attraction effects. Previously, \citeand{BockEberhard93} tested whether singular collective nouns as attractors may induce agreement errors, like plural non-collective nouns. Their results suggested that collective nouns are not effective attractors and notional plurality do not interfere with the subject-verb dependency in English. However, \citeand{HaskellMacDonald2003} used collective nouns as agreement controllers in their experiment and tested whether semantic plurality on the controller affected the percentage of attraction errors. They manipulated the type of the head (collective x non-collective) and the number marking on the attractor (plural x singular). The head noun was always singular. One set of experimental items can be found in (\ref{ex:hask03}). They conducted a production experiment with a sentence completion task. They accompanied their production experiment with offline grammaticality judgments in the following experiment. 

\ea \label{ex:hask03}
  \ea {Non-collective Head}\\*The {actor} in the weekend {performance/performances} \ldots{} 
  \ex {Collective Head}\\*The {cast} in the weekend {performances/performances} \ldots{}  
  \z
\z

Their results suggested a significant main effect of collective heads. Participants made more agreement errors when the agreement controller was notionally plural. There was also the main effect of the plural attractor. Independent of the head type, participants completed the preambles with erroneously marked verb when there was a plural attractor present. Moreover, there was also a significant interaction between the collective controllers and the plural attractor. These findings suggested that collective nouns affected the percentage of agreement errors when they were the subject heads. The semantics of the head noun interacted with the grammatical number feature. Again, their results contradicted the predictions of the Feature Percolation account. 

Considering these findings that cannot be explained via the Feature Percolation account, Bock, Eberhard, Cutting, Meyer, and Schriefers (2001) proposed and \citeand{EberhardEtAl2005} refined an account of agreement attraction where they divide the attraction phenomenon into two processes: Conceptualization (Marking) and Grammatical Encoding (Morphing), thus called Marking and Morphing account. While Marking deals with the notional number and its reflection to the syntax in the form of features, Morphing is concerned with the representation formed in morpho-phonological encoding. In their account, there are two sources of number information: semantic and syntactic; in other words, notional and inflectional. With different degrees and constraints on them, both can influence the subject-verb agreement. 

There are two critical assumptions in Marking and Morphing account. Firstly, the number value is not binary, but a continuum. In addition to unambiguously plural and singular nouns, represented with 0 and +1 values in the continuum, respectively, there might be nouns, NPs, or DPs whose number is not strictly clear. For example, consider the subject \emph{each} in (\ref{ex:ambiguous_each}).\footnote{I would like to thank Elena Guerzoni for her judgments with respect to sentences with a subject containing the word \emph{each}.}

\ea \label{ex:ambiguous_each} Each was/were repairing the car. \z

The word \emph{each} is ambiguous here; thus, the marking on the verb can be either plural or singular. In contexts that license distributive readings where each person on their own tried to repair the car, the singular verb is preferred. On the other hand, the plural verb is preferred if our context licenses the reading where people are trying to repair the car altogether. Like the word \emph{each}, the words that are ambiguous in their numbers are represented with a value that falls between 0 and +1. 

In addition to ambiguities stemming from the interaction of lexical meaning and context, other ambiguities may arise from notional number information of a word or other mismatching number markings in the sentence. For example, the word \emph{gang} is notionally plural; thus, it is not unambiguously singular or plural. In addition, the phrase \emph{the key to the cabinets} is also unambiguous in number. Even though the head is grammatically and notionally singular, other nouns in the vicinity have a mismatching number marking. According to the Marking and Morphing account of agreement attraction, presence of other nouns with a mismatching number contributes to the number uncertainty.

The second assumption is related to how to integrate different sources of number information and how the final number representation will be calculated. To this end, they utilize the spreading activation formula given in (\ref{eq:dell1986}) \citep{Dell1986}. This formula extends from the works that saw language comprehension as a constraint satisfaction problem \citep{TrueswellTanenhaus94}. To solve problems during the language comprehension, they offered a framework where soft stochastic constraints that might vary in their importance are satisfied, and the result of a processing is the interaction of these constraints. Marking and Morphing theory uses the function \citeand{Dell1986} introduced and implemented for phonological encoding and the spread of phonological features \citep{Dell88}. In short, the formula below sums the notional number of the head noun \emph{(S(n))} and the weighted sum of other pieces of number information \emph{(S(m))} in the sentence. The final product is the conceptual number \emph{(S(r))}. The additional pieces of number information are weighted using syntactic information. Their relative syntactic distance to the root node of the subject phrase will be used as a weight.

\begin{equation} \label{eq:dell1986}
  S(r){\ }={\ }S(n){\ }+{\ }\sum_{j}\Big(w_j{\ }\times{\ }S(m)_j \Big)
\end{equation}

When \emph{(S(r))}, which is the only available number value from the equation to the agreement mechanisms, falls somewhere between 0 and 1, the Marking and Morphing account claims that participants may interpret this number information as ambiguous. As a result, they may form plural representations --- multiple \emph{keys} instead of a single \emph{key} in our case --- which would result in participants making agreement errors in production or finding ungrammatical sentences with plural attractors occasionally grammatical.  

Following the equation (\ref{eq:dell1986}), we can infer that the notional number of the head noun directly affects the final number representation. Previous studies summarized in this section verify this prediction, and the fact that notional number information from other sources does not contribute to the final representation can also be retrieved from the equation. 

Additionally, we would expect that hierarchically lower plural information would have less impact than hierarchically higher plural marked elements. For example, when there are two prepositional phrases embedded recursively as in (\ref{ex:franck02}), the syntactically higher element, \emph{presidents}, creates higher interference compared to the more local but syntactically lower element, \emph{companies}. Similarly, the elements embedded in a relative clause are expected to induce fewer agreement errors than atracctors embedded in prepositional phrases. Both predictions of Marking and Morphing are formalized in the formula and verified by experimental findings. 

Moreover, according to the Marking and Morphing accounts, participants should have similar difficulties in detecting sentence acceptability in ungrammatical and grammatical sentences when a plural attractor is present. Consider the sentences in (\ref{ex:ch2_asym}). 

\ea \label{ex:ch2_asym}
  \ea[]{\label{ex:ch2_plsg}The {key} to the {cabinets} is rusty.}
  \ex[*]{\label{ex:ch2_plpl}The {key} to the {cabinets} are rusty.}
  \z
\z

In the account specified above, the final number representation is only determined by the information provided before the verb. This suggests that the number marking on the verb should not play any role. Participants should have fewer accurate answers in both sentences compared to their singular attractor counterparts. Apart from \citeand{PearlmutterGarnseyBock:1999}, many studies conducted in number agreement attraction showed that this prediction did not hold \citep[See][for an overview]{HammerlyEtAl2019}. However, a recent study by \citeand{HammerlyEtAl2019} showed that participants in these studies had an a priori response bias towards giving yes responses, which amplify the attraction effects in ungrammatical sentences, but significantly decreases the effect of plural attractor in grammatical sentences.



\section{Cue-based retrieval account} \label{sec:ret}

The theories up until this point is typically referred as representational \citep{HammerlyEtAl2019} or encoding \citep{AvetisyanEtAl:2020} accounts due to their focus on the representations and encodings in agreement attraction effects. Their formulation of agreement attraction solely depend on a single assumption: attraction results from a faulty representation of the agreement controller and the attractor. 

However, additional experimental work showed that this assumption and the tenets of the Marking and Morphing account could not explain all factors that impact agreement attraction findings. Some of these factors that cannot possibly be explained with the Marking and Morphing account include the effects of the verb number, linear distance, and the presence of clause-external attractors. 

For instance, as discussed recently, the Marking and Morphing account expects a symmetrical attraction effects in grammatical and ungrammatical sentences. In comprehension, participants should exhibit grammaticality illusion and ungrammaticality illusion. That is, they should be illusioned to think that an ungrammatical sentence is grammatical (grammaticality illusion) and vice versa. However, \citeand{WagersEtAl:2009} found that participants exhibit only grammaticality illusion but not ungrammaticality illusion in reading experiments. Five of the seven experiments presented in their work showed no effect of plural attractor in grammatical sentences. Their experiments included two structures (PP and RC) and two experimental frameworks (self-paced reading and speeded acceptability judgment). 

In their Experiment 4, most sentences were based on \cites{PearlmutterGarnseyBock:1999} experimental sentences. They only manipulated the number of the attractor and the verb (plural x singular), the subject head's number did not change within conditions. One set of experimental conditions can be found in (\ref{ex:wagersch2}). Following from their previous experiments in the same study, they hypothesized that the difference in acceptability should only be observable between (\ref{ex:wagersch2plpl}) and (\ref{ex:wagersch2sgpl}), but not between (\ref{ex:wagersch2plsg}) and (\ref{ex:wagersch2sgsg}).

\ea \label{ex:wagersch2}
  \ea[*]{\label{ex:wagersch2plpl}{Plural Attractor \& Ungrammatical (Plural Verb)} \\* The {key} to the {cells} unsurprisingly were rusty from many years of disuse.}
  \ex[]{{\label{ex:wagersch2plsg}Plural Attractor \& Grammatical (Singular Verb)} \\* The {key} to the {cells} unsurprisingly was rusty from many years of disuse.}
  \ex[*]{{\label{ex:wagersch2sgpl}Singular Attractor \& Ungrammatical (Plural Verb)} \\* The {key} to the {cell} unsurprisingly were rusty from many years of disuse.}
  \ex[]{{\label{ex:wagersch2sgsg}Singular Attractor \& Grammatical (Singular Verb)} \\* The {key} to the {cell} unsurprisingly was rusty from many years of disuse.}
  \z
\z

They found that Experiment 4's results were comparable with their previous experiments in the same study, and their results were not due to the RC structure they used in the previous experiments. Unlike \cites{PearlmutterGarnseyBock:1999} findings and the Marking and Morphing account predictions, participants did not exhibit additional processing difficulty in grammatical sentences with plural attractors, or there was no difference in acceptability in grammatical condition pair. This phenomenon called grammaticality asymmetry suggests that the attraction effects are not due to the malformed representations of determiner phrases. The number marking on the verb or the grammaticality of the sentence also has a say in the attraction effects. Even though this asymmetry was replicated many times previously (Lago, Acu{\~n}a-Fari{\~n}a, \& Meseguer, 2021), a recent study by \citeand{HammerlyEtAl2019} argued that this asymmetry is a residue of participants' response bias, which we discuss in Chapter \ref{ch:exp3}.

In addition to grammaticality asymmetry, \citeand{WagersEtAl:2009} found that clause-external elements may induce attraction effects, which cannot be accounted for with the Marking and Morphing account. In Experiment 2, a self-paced reading experiment, they used experimental sentences with object relative clauses as in (\ref{ex:wagersch2exp2}). They manipulated the number of the embedded verb and the attractor (plural x singular). They used the RC head as an attractor. 

\ea \label{ex:wagersch2exp2}
  \ea[*]{\label{ex:wagersch2exp2plpl}{Plural Attractor \& Ungrammatical (Plural Verb)} \\* The {musicians} who the {reviewer} praise so highly will probably win a Grammy.}
  \ex[]{{\label{ex:wagersch2exp2plsg}Plural Attractor \& Grammatical (Singular Verb)} \\* The {musicians} who the {reviewer} praises so highly will probably win a Grammy.}
  \ex[*]{{\label{ex:wagersch2exp2sgpl}Singular Attractor \& Ungrammatical (Plural Verb)} \\* The {musician} who the {reviewer} praise so highly will probably win a Grammy.}
  \ex[]{{\label{ex:wagersch2exp2sgsg}Singular Attractor \& Grammatical (Singular Verb)} \\* The {musician} who the {reviewer} praises so highly will probably win a Grammy.}
  \z
\z

They found that participants read the region following the verb (\emph{so}) faster in ungrammatical sentences with plural attractors (\ref{ex:wagersch2exp2plpl}) than in ungrammatical sentences with singular attractors (\ref{ex:wagersch2exp2sgpl}). The reading times of the same region in grammatical conditions were not substantially different from the ungrammatical condition with a plural attractor, and there were no meaningful difference between the subconditions in grammatical sentences.

The most important aspect of their findings was the magnitude of the attraction. Previous attraction accounts would predict diminished effect in magnitude when the attractor has increased syntactic distance to the head. However, attraction effects in \cites{WagersEtAl:2009} findings with RC and PP were comparable. Neither the Feature Percolation account, which does not allow downward percolation, nor the Marking and Morphing account, which allows downward percolation but weights number information according to the syntactic distance, was able to explain these findings, which are consistently shown in more than a single experiment.

Moreover, \citeand{HaskellMacDonald2005} tested whether the linear distance between the agreement controller and the verb influences the agreement attraction effects. They conducted a production experiment with a sentence-completion task using preambles in (\ref{ex:haskell05}). Their preambles were in the form of a yes-no question. The agreement controller, consisting of two disjuncts, one of which is plural, was in an embedded phrase headed by the complementizer \emph{if}. They manipulated which noun would be plural in their experimental conditions.

\ea \label{ex:haskell05}
  \ea {SP Configuration} \\* Can you ask Brenda if {the boy or {the girls}} \ldots{}?
  \ex {PS Configuration} \\* Can you ask Brenda if {{the boys} or the girl} \ldots{}?
  \z
\z

They have found that the participants made fewer agreement errors in the PS configurations where the attractor is not immediately before the to-be-produced verb. They assumed that disjuncts in coordinating structures do not differ in their syntactic distance to the verb and can be represented with a ternary branching. With this assumption in mind, they interpreted their results as evidence of an effect of linear distance independent of syntactic depth or distance difference. Even if we assume \cites{Progovac98} asymmetric conjunction analysis ([the boy(s)] [or the girl(s)]) which borne out of binding facts of English, \cites{HaskellMacDonald2005} findings suggest that a syntactically closer plural DP (the boys) induces fewer agreement errors than a more deeply embedded plural DP (the girls).

Considering these additional findings, another account of agreement attraction gained more visibility: the cue-based retrieval account \citep{WagersEtAl:2009, LagoEtAl2015}. Retrieval theories claim the minimal unit comprehenders deal with is an information structure called \emph{chunks} \citep{LewisVasishth:2005}. Participants encode and store relevant features of words into chunks, such as [+{subj}] and [+{pl}]. These features are later used to retrieve the controller of a dependency, in our case, the subject head. The retrieval process is triggered by the probe of the dependency, the verb in the subject-verb agreement. This process is driven by the cues specified by the probe. 

For example, English verbs may include cues for the number, case, and syntactic position \citep{ArnettWagers:2017}. When there is an element that fully matches the cues provided by the probe, this chunk is retrieved from the memory and utilized in the processing. However, when there is more than a single match for the given cues due to cue overlap, interference may surface; a distractor element may interfere with the dependency (J\"ager, Mertzen, Van Dyke, \& Vasishth, 2020). Interference may also occur when no element fully matches the cues, but when multiple elements partially match the cues necessary to satisfy a dependency. 

Consider the canonical example `\emph{The key to cabinets is rusty.}' According to the cue-based retrieval account of agreement attraction \citep{WagersEtAl:2009}, the chunk for the controller \emph{key} contains the features [+{subj}] and [+{sg}], while the attractor \emph{cabinets} is abstracted with the features [-{subj}] and [+{pl}]. When participants read the first two DPs, they encode these features into chunks and store the chunks. Upon reading the verb \emph{is}, a search begins with the specified cues by the verb: [+{subj}] and [+{sg}]. We have a single full match in this example: the controller \emph{key}. Since we will have a single complete match even when the attractor is singular, the cue-based retrieval account predicts that there should be no differences in the acceptability rates of these sentences.

However, when we have a sentence like `\emph{* The key cabinets are rusty,}' there is no single full match. The verb provides the cues [+{subj}] and [+{pl}]. The supposed controller \emph{key} only matches the cue [+{subj}] but does not satisfy the other feature concerning the number. Similarly, the attractor \emph{cabinets} only matches the cue [+{pl}] but does not satisfy the subjecthood cue. Since no element fully matches the cues, but multiple elements partially match the cues, an interference may surface. On some occasions, participants may retrieve the attractor \emph{cabinets} instead of the controller, which results in increased acceptability of the ungrammatical sentence with plural attractors. 

With its singular attractor counterpart, the increase in acceptability/error rates is not expected since the attractor \emph{cabinet} will not match with either subjecthood ([+{subj}]) or  the number [+{pl}]) related cues. Participants will only entertain the word \emph{key} as a controller in single attractor conditions even though it does not fully match the cues.

In essence, the cue-based retrieval theory formalizes attraction errors because of a misretrieval in the case of possible interference. Unlike the Feature Percolation and Marking and Morphing accounts, the process of forming representations, or encoding features into a chunk, is not the source of attraction. The accounts that explain agreement attraction as a retrieval problem assume that this process is error-free. However, the real culprit is the retrieval process.

By integrating the role of memory and retrieval, cue-based accounts could explain grammaticality asymmetry easily. In this account, the attraction is only expected to arise in the case of ungrammaticality, lack of a single complete match. Since grammatical sentences satisfy the dependency with a full match, no interference is created by the presence of attractors. It also explains the attraction effects induced by clause-external attractors. Since there is no reference to the structural relation between the attractor and the controller, it does not matter where the attractor resides syntactically. 



\section{Agreement attraction in Turkish}

In the previous sections of this chapter, we have covered significant accounts for agreement attraction effects. We also covered some influential experiments that led to these accounts. These experiments were conducted in English, Italian, French, Dutch, and French. In addition to these languages, attraction effects --- not only number but also gender, case, and honorific attraction --- were found to be robust in Arabic \citep{TuckerEtAl:2015}, Eastern Armenian \citep{AvetisyanEtAl:2020}, Greek \citep{PaspaliMarinis2020}, Hebrew \citep{DeutschDank2011}, Hindi \citep{BhatiaDillon:2020}, Korean \citep{KwonSturt2016}, Russian (Lorimor, Bock, Zalkind, Sheyman, \& Beard, 2008; Slioussar, 2018; Slioussar \& Malko, 2016), Slovak \citep{BadeckerKuminiak2007}, Spanish \citep{LagoEtAl2015, LagoEtAl2021}, and Turkish \citep{LagoEtAl2019}. In this section, we will cover the experimental findings on agreement and number in Turkish. To our knowledge, there has been three studies covering this interaction: \citeand{LagoEtAl2019}, \citeand{Ozay20}, and Ayg\"une\c{s}, Ka\c{s}{\i}k\c{c}{\i}, Ayd{\i}n, and Demiralp (2021) which is based on \citeand{Aygunes13}. 

\subsection{Ayg\"une\c{s} et al. (2021)}


\citeand{AygunesEtAl21} investigated the processing difference between number features and person features in Turkish using Event Related Potentials (ERP). They used personal pronouns as subjects and manipulated the the number and the person of the subject (\Fsg{} x \Fpl{} x \Ssg{} x \Spl{}). They also manipulated the grammaticality of the sentence: (i) grammatical, (ii) ungrammatical due to person-wise mismatching marking, (iii) ungrammatical due to number-wise mismatching marking, or (iv) ungrammatical due to both person and number-wise mismatching marking. As a result, their experiment had 16 conditions, represented in (\ref{ex:ay21}). Their experimental items consisted of three words and did not include any attractors.

\ea \label{ex:ay21}
    \gll {Ben/Biz/Sen/Siz} yeme\u{g}-i yap-t{\i}-{m/n/k/n{\i}z}.\\
    {\Fsg/\Fpl/\Ssg/\Spl} meal-\Acc{} make-\Pst-{\Fsg/\Fpl/\Ssg/\Spl}.\\
    \glt `{I/We/You\textsubscript{\it sg}/You\textsubscript{\it pl}} prepared\textsubscript{\Fsg/\Fpl/\Ssg/\Spl} the meal.'
\z

They contrast N400 + P600 patterns in the processing of ungrammatical sentences due to mismatching person feature, mismatching number feature, or both simultaneously. Their results show that mismatching person-related features induce greater N400 effects compared to mismatching number-related features. They interpreted their results as evidence of a structural difference between the person and number features. 

\subsection{\"Ozay (2020)}

\citeand{Ozay20} examined the processing difficulty of the agreement phenomenon in different syntactic structures. They used DPs modified with a numeral as the subject of the sentence and only manipulated the person marking on the verb (\Tsg{} x \Tpl{} x \Spl{} x \Fpl) as in (\ref{ex:oz20}). 

%Since numeral constructions in Turkish does not marked with any plural marking form-wise they matched \Tsg{}-conditions, but feature-wise experimental items matched \Tpl{}-conditions. 

\ea \label{ex:oz20}
    \gll \"U\c{c} ki\c{s}i bu havuz-da y\"uz-\"uyor-{du/lar-d{\i}/du-nuz/du-k} daha ge\c{c}en hafta.\\
    three person this pool-\Loc{} swim-\Impf-{\Pst.\Tsg/\Tpl-\Pst/\Pst-\Spl/\Pst-\Fpl} just last week\\
    \glt `Three people were swimming in this pool just last week.'\\*
    `They/You/We were swimming in this pool just last week as a group of three people.'
\z

Their findings suggest that participants had the most difficulty reading the main verb in \Spl{} and \Tpl{} conditions. There was no substantial difference between the reading times of these two conditions. The least difficult sentences for participants were the ones with \Tsg{} conditions. They argue that these findings conflict with previous agreement attraction findings. Since \Tpl{} (match) conditions read slower than \Tsg{} (mismatch) conditions.

Their experimental conditions are not directly comparable. They say that \Tsg{} and \Tpl{} can be used in the same environment with no semantic or syntactic change whereas \Fpl{} and \Spl{} conditions are syntactically more complex as in (\ref{ex:oz20_funky}). However, pro-drop scenarios are one of the most frequent environments for using \Tpl{}-marking. It is possible that participants speculated on complex syntactic structures after seeing other conditions. Thus, we believe that \Tpl{} and \Tsg{} conditions are not match and mismatch conditions. Rather, they are different structures, and \Tpl{} is both syntactically and semantically closer to \Fpl{} and \Spl{}.

\ea \label{ex:oz20_funky}
    \gll \emph{pro\textsubscript{\it i}} [\textsc{pro}\textsubscript{\it i} \"u\c{c} ki\c{s}i (olarak)] y\"uz-\"uyor-du-k.\\ 
    \emph{pro\textsubscript{\it i}} [\textsc{pro}\textsubscript{\it i} three person (be)] swim-\Impf-\Pst-\Fpl{}\\
    \glt `We were swimming as a group of three people.'
\z

\subsection{Lago et al. (2019)}

Another study conducted on Turkish number agreement is \cites{LagoEtAl2019} study. Their study tested whether Turkish native and heritage speakers exhibit agreement attraction effects in a speeded acceptability judgment experiment with sentences like (\ref{ex:lagoch2}). They manipulated the number on the verb and the attractor (plural x singular). The attractor was a genitive marked nominal modifier, a possessor, preceded the head. In this thesis, we only focus on the results of native Turkish speakers. \cites{LagoEtAl2019} study is the only study in Turkish that investigated the interaction between number and agreement with the use of attractors. 

\ea \label{ex:lagoch2}
    \ea[*]{{Plural Attractor, Ungrammatical (Plural Verb)} \label{ex:lagoch2-plpl}\\*
    \gll {\c{S}ark{\i}c{\i}-lar-{\i}n} {vokalist-i} sahne-de s\"urekli z{\i}pla-d{\i}-lar.\\
    singer-\Pl-\Gen{} backup-\Poss{} stage-\Loc{} non-stop jump-\Pst-\Tpl{}\\
    \glt `The singers' backup vocalist jumped\textsubscript{\Pl{}} on stage non-stop.'}
    \ex[]{{Plural Attractor, Grammatical (Singular Verb)} \label{ex:lagoch2-plsg} \\*
    \gll {\c{S}ark{\i}c{\i}-lar-{\i}n} {vokalist-i} sahne-de s\"urekli z{\i}pla-d{\i}.\\
    singer-\Pl-\Gen{} backup-\Poss{} stage-\Loc{} non-stop jump-\Pst{}\\
    \glt `The singers' backup vocalist jumped\textsubscript{\Sg{}} on stage non-stop.'}
    \ex[*]{{Singular Attractor, Ungrammatical (Plural Verb)} \label{ex:lagoch2-sgpl}\\*
    \gll {\c{S}ark{\i}c{\i}-n{\i}n} {vokalist-i} sahne-de s\"urekli z{\i}pla-d{\i}-lar.\\
    singer-\Gen{} backup-\Poss{} stage-\Loc{} non-stop jump-\Pst-\Tpl{}\\
    \glt `The singer's backup vocalist jumped\textsubscript{\Pl{}} on stage non-stop.'}
    \ex[]{{Singular Attractor Grammatical (Singular Verb)} \label{ex:lagoch2-sgsg}\\*
    \gll {\c{S}ark{\i}c{\i}-n{\i}n} {vokalist-i} sahne-de s\"urekli z{\i}pla-d{\i}.\\
    singer-\Gen{} backup-\Poss{} stage-\Loc{} non-stop jump-\Pst{}\\
    \glt `The singer's backup vocalist jumped\textsubscript{\Sg{}} on stage non-stop.'}
    \z
\z

Previous attraction studies showed that possessors do not induce attraction effects in English and genitive-marked DPs are not robust attractors (Nicol, Barss, \& Barker, 2016). In their research, \citeand{NicolEtAl:2016} found that the preambles like `\emph{The {elves'} house with the tiny window \ldots{\ }}' did not give rise to additional agreement errors compared to their singular attractor counterparts with the word \emph{elf's}. They argued that the lack of attraction with a possessor as an attractor was because the possessor carried an overt marking that signalled that they are not heads or subjects. Similarly, \citeand{LagoEtAl2019} argued that since genitive heads could be subjects in embedded sentences, genitive-marked modifiers might be good candidates for being an attractor. Since their form is compatible with subjecthood, they may induce attraction effeects in Turkish. 

Their results showed that the overall acceptability was not affected by the number of the attractor in grammatical sentences. However, the acceptability of ungrammatical sentences was sensitive to the presence of a plural attractor. Their results were comparable with the previous findings of agreement attraction and the grammaticality asymmetry. Thus, they interpreted their results as evidence for a cue-based retrieval account. They argued that attraction occurred due to an error-driven process in which participants erroneously retrieved the attractor rather than the head only when there was an agreement error present. This understanding of attraction supports the discrepancy between the grammatical and ungrammatical sentences' acceptability.

Their results also pointed out that the case information or the form of the case information is an important feature that has a role in the computation of agreement. The fact that genitive-marked nouns did not induce agreement attraction in English, but in Turkish showed that the function of a case is also an important cue in addition to the exact specifications of a case. In addition to the case features like [+{gen}] or [+{nom}], function-related features like [+{subj}], [+{obj}], or [+{obl}] must be specified. 



\section{Role of case syncretism in agreement attraction} \label{sec:sync}

Previous psycholinguistics studies showed that the information provided with the overt or abstract cases play a vital role in the processing (\"Ozge, K\"untay, \& Snedeker, 2019; Yamashita, 1997; Kim, 1999; Loga\v{c}ev \& Vasishth, 2012; Babyonyshev \& Gibson, 1999). 

For instance, \citeand{BabyGibson99} and Fedorenko, Babyonyshev, and Gibson (2004) tested how case marking affects the processing of center-embedding sentences in Japanese and Russian. \citeand{BabyGibson99} asked participants to rate the complexity of the sentences such as (\ref{ex:baby99}). They manipulated the transitivity of the verb (intransitive x transitive) and the case marking on the most-outer subject (topic marker -wa x nominative marker -ga). Subjects in Japanese can be optionally marked with a topic marker to deliver certain pragmatic and semantic meanings. They utilized this feature as a manipulation in their experiment. As for the rest of the subjects, they were always marked with the nominative case. 

\ea \label{ex:baby99}
  \ea {Intransitive, No Topic Marker} \\*
  \gll Wakai kyooju-{ga} [TA-ga [gakusei-ga konransita to] sengensita to] utagatta.\\
  young professor-{\Nom{}} [teaching\_assistant-\Nom{} [students-\Nom{} panicked that] announced that] doubted\\
  \glt `The young professor doubted that the teaching assistant announced that the students panicked.'
  \ex {Transitive, No Topic Marker }\\*
  \gll Kankyaku-{ga} [rajioanaunsaa-ga [yuumenia sukeetosensyu-ga sukeetogutu-o kowasita to] sengensita to] utagatta.\\
  spectator-{\Nom{}} [radio\_announcer-\Nom{} [famous skater-\Nom{} skate-\Acc{} broke that] announced that] doubted\\
  \glt `The spectatour doubted that the radio announcer announced that the famous skater broke a skate.'
  \ex {Intrantisitve, Topic Marker}\\*
  \gll Eegakantoku-{wa} [purodyusaa-ga [kireina joyuu-ga koronda to] itta to] omotteiru.\\
  film\_director-{\Top{}} [producer-\Nom{} [pretty actress-\Nom{} fell that] said that] thinks\\
  \glt `As for the film director, he thinks that the producer said that the pretty actress fell.'
  \ex {Transitive, Topic Marker}\\*
  \gll Ounaa-{wa} [sihainin-ga [kyaku-ga wazato ueitaa-o osita to] itta to] omotteiru.\\
  owner-{\Top{}} [manager-\Nom{} [guest-\Nom{} deliberately waiter-\Acc{} pushed that] said that] thinks\\
  \glt `As for the owner he thinks that the manager said that a customer deliberately pushed the waiter.'
  \z
\z

Their results suggested that participants found sentences where the most-outer subject is marked with the nominative case harder to understand and marked those sentences more complex. When there is a mismatching case-marking, the processing center embeddings were relatively easy. They interpreted their results as evidence for retrieval interference: as the number of candidates with the same specifications increases, the interference effect also increases, which an be seen as increased perceived complexity. 

In a subsequent experiment, \citeand{FedorenkoEtAl2004} tested whether or not the effects of case marking are due to abstract or phonological case marking. \cites{BabyGibson99} findings were not clear whether the findings are due to the the difference in form or difference in abstract case. They conducted a self-paced reading experiment with comprehension questions after every item. They utilized the syncretism between the accusative case with feminine nouns and the dative case with masculine nouns. As seen in (\ref{ex:fed04-femacc}) and (\ref{ex:fed04-mascdat}), both are marked with the \emph{-u} ending while the accusative case surfaces as \emph{-a} with masculine nouns and the dative case surfaces as \emph{-e} with feminine nouns.

\ea \label{ex:fed04}
  \ea \label{ex:fed04-femacc} {Abstract Case \& Form Match}\\*
    \gll [[Uva\v{z}av\v{s}uju skripa\v{c}k{u}] pianistk{u}] razozlil diri\v{z}er iz izvestnoj konservatorii posle generalnoj repetitsii.\\
    [[respecting violinist.\F{}.\Acc{}] pianist.\F{}.\Acc{}] angered conductor.\Nom{} from famous conservatory after final rehearsal\\
    \glt `After the final rehersal, the conductor from a famous conservatory angered the pianist\textsubscript{\F{}.\Acc{}} who respected the violinist\textsubscript{\F{}.\Acc{}}.'
  \ex \label{ex:fed04-mascacc} {Abstract Case Match \& Form Mismatch}\\*
  \gll [[Uva\v{z}av\v{s}uju skripa\v{c}k{a}] pianistk{u}] razozlil \ldots{}\\
  [[respecting violinist.\M{}.\Acc{}] pianist.\F{}.\Acc{}] angered \ldots{}\\
  \glt `After the final rehersal, the conductor from a famous conservatory angered the pianist\textsubscript{\F{}.\Acc{}} who respected the violinist\textsubscript{\M{}}.\Acc{}.'
  \ex \label{ex:fed04-mascdat} {Abstract Case Mismatch \& Form Match}\\*
  \gll [[Pozvoniv\v{s}uju skripa\v{c}k{u}] pianistk{u}] razozlil \ldots{}\\
  [[having\_called violinist.\M{}.\Acc{}] pianist.\F{}.\Acc{}] angered \ldots{}\\
  \glt `After the final rehersal, the conductor from a famous conservatory angered the pianist\textsubscript{\F{}.\Acc{}} who had called the violinist\textsubscript{\M{}.\Acc{}}.'
  \ex \label{ex:fed04-femdat} {Abstract Case \& Form Mismatch}\\*
  \gll [[Pozvoniv\v{s}uju skripa\v{c}k{e}] pianistk{u}] razozlil \ldots{}\\
  [[having\_called violinist.\F{}.\Dat{}] pianist.\F{}.\Acc{}] angered \ldots{}\\
  \glt `After the final rehersal, the conductor from a famous conservatory angered the pianist\textsubscript{\F{}.\Dat{}} who had called the violinist\textsubscript{\F{}.\Acc{}}.'
  \z
\z

Their results suggested that neither the phonological form of the case nor the abstract case feature does not alone induce interference effects. Participants read {Abstract Case \& Form Match} conditions significantly more slowly, and their accuracy was significantly lower in the same conditions. However, the rest of the experimental conditions showed no substantial difference both in their reading times and response accuracies. 

However, the effect of case marking is not clear in agreement attraction literature. The first study that tackled this question was \cites{HatsuikerEtAl2001} production experiment, where they instructed participants to complete sentence preambles. They manipulated the type of the attractor (NP x pronoun), the number of the attractor (plural x singular), and the pronoun ambiguity (case-ambiguous pronoun x unambiguous pronoun). To manipulate pronoun ambiguity, they used inanimate nouns. The inanimate Dutch plural pronoun is ambiguous between the accusative and nominative case marking, whereas the other pronouns are unambiguously accusative. One set of example conditions can be seen in (\ref{ex:hart_pro}). 

\ea \label{ex:hart_pro}
  \ea \label{ex:hart_pro_npunamb} {Full NP \& Animate}\\*
    \gll Ed ziet dat de {kapitein} de {zeerover(-s)} \ldots{} \\
    Ed sees that the captain the pirate(-\Pl{}) \ldots{} \\
    \glt `Ed sees that the captain \ldots{}$_{verb}$ the pirate(s).'
  \ex \label{ex:hart_pro_prounamb} {Unambiguous Pronoun \& Animate}\\*
    \gll Ed ziet dat de {kapitein} {hem/hen} \ldots{} \\
    Ed sees that the captain him/hem \ldots{} \\
    \glt `Ed sees that the captain \ldots{}$_{verb}$ him/them.'
  \ex \label{ex:hart_pro_npamb} {Full NP \& Inanimate}\\*
    \gll Tanja zegt dat de {verkoper} de {auto(-s)} \ldots{} \\
    Tanja says that the salesman the car(-\Pl{}) \ldots{} \\
    \glt `Tanja says that the salesman \ldots{}$_{verb}$ the car(s).'
  \ex \label{ex:hart_pro_proamb} {Ambiguous Pronoun \& Inanimate}\\*
    \gll Tanja zegt dat de {verkoper} {hem/ze} \ldots{} \\
    Tanja says that the salesman  him/them \ldots{} \\
    \glt `Tanja says that the salesman \ldots{}$_{verb}$ it/them.'
  \z
\z

Their results suggested that ambiguous pronouns led participants to make more agreement errors than unambiguous pronouns in number mismatching conditions. The presence of the ambiguous pronoun \emph{ze} resulted in as many agreement errors as the conditions with full NPs. When there was an overt/unambiguous case marking, participants made substantially fewer errors. 

Another study that used a sentence completion framework and used pronouns was \cites{NicolAnton2009} study. They conducted their experiment in English with preambles such as `\emph{The {bill} from account(s)/him(them)} \ldots{\ }'. They aimed to test the effects of overt case marking in English, which was only possible with pronouns like Dutch. Unlike \citeand{HatsuikerEtAl2001}, \citeand{NicolAnton2009} only manipulated the number of the attractor (singular x plural) and the type of the attractor (NP x pronoun). Similar to \cites{HatsuikerEtAl2001} findings, they found that overt case-marking (the use of pronouns) diminished the error rates in subject-verb agreement. 

However, both of these studies could not differentiate between the effects of pronoun use and the effects of overt case-marking. In a subsequent production study with a sentence completion task, Hartsuiker, Schriefers, Bock, and Kikstra (2003) tested the effects of overt-case marking in a language that uses case-marking with noun phrases: German. They have utilized ambiguous case markings and article forms in their experiment. 

The case information has a morphophonological reflex both on the article and the noun in German. For example, the singular noun \emph{man} is \emph{der Mann} in the nominative and \emph{dem Mann} in the dative, whereas the plural noun \emph{men} is \emph{die M\"anner} in the nominative and \emph{den M\"annern} in the dative. Another important characteristic of German is that some case marking may surface in an ambiguous form. While the noun \emph{man} is unambiguously dative-marked in \emph{dem Mann}, its surface of a plural and nominative-marked \emph{men}, \emph{die M\"anner}, is ambiguous between the accusative and nominative forms. Even though this specific ambiguity is limited to plural forms with masculine nouns, a similar syncretism can also be observed in singular forms with feminine and neuter nouns. For example, the article of the word \emph{Demonstration} in German is not ambiguous between a singular and a plural form when the noun is marked with a dative case. However, the nominative and the accusative forms of this noun's article surfaces as \emph{die} independent of the case and the plurality. One set of examples in which this ambiguity is utilized by \citeand{HartsuikerEtAl2003} is provided in (\ref{ex:hart03}). They manipulated the number of the attractor (plural x singular) and the case ambiguity on the attractor (unambiguously dative x ambiguous between nominative and accusative) by changing the preposition. 

\ea \label{ex:hart03}
  \ea {Unambiguously Dative}\\*
    \gll Die {Stellungnahme} zu der/den {Demonstration(-en)} \ldots{} \\
    the.\F.\Nom.\Sg{} position on the.\F.\Dat.\Sg/\Pl{} demonstration(-\Pl) \ldots{} \\
    \glt `The position on the demonstrations(s) \ldots{}'
  \ex {Ambiguous between Nominative and Accusative}\\*
    \gll Die {Stellungnahme} gegen die {Demonstration(-en)} \ldots{} \\
    the.\F.\Nom.\Sg{} position against the.\F.\Nom/\Acc.\Sg/\Pl{} demonstration(-\Pl) \ldots{} \\
    \glt `The position against the demonstrations(s) \ldots{}'
  \z
\z

Their results were comparable with the previous findings on the effects of overt-case marking in the agreement attraction phenomenon: unambiguous case markings reduced the overall agreement errors done in number mismatching conditions. While people still made agreement errors with unambiguous conditions in which the noun is marked with the dative case, they made significantly more errors in conditions with ambiguously marked nouns. This finding verified that their previous results were not solely due to the word category difference (noun vs. pronoun).

Another language in which the effect of case was investigated was French. \citeand{FranckEtAl2006} conducted a production experiment with a sentence completion task. Participants were provided with a sentence preamble and a verb and asked to complete the sentence correctly. They have manipulated the number of the head and the attractor (plural x singular) and the type of the attractor (preverbal object clitic x prepositional subject modifier). While the prepositional subject modifier is syncretic between the nominative and the accusative case, the preverbal object clitic is distinctive in its case marking. One set of experimental conditions can be found in (\ref{ex:franck2006_exp2}).

\ea \label{ex:franck2006_exp2}
  \ea {Preverbal Object Clitic}\\*
  \gll Le(-s) {professeur(-s)} {le(-s)} \ldots{} \\
  the(-\Pl) professor(-\Pl) it.\Acc(-\Pl) \ldots{} \\
  \glt `The professor \ldots$_{verb}$ it/them.'
  \ex {Prepositional Subject Modifier}\\*
  \gll Le(-s) {professour(-s)} {de l'\'el\`eve/des \'el\`eves} \ldots{} \\
  the(-\Pl) professor(-\Pl) {of the student/the students} \ldots{} \\
  \glt `The professor of the student(s) \ldots{}$_{verb}$'
  \z
\z

Their results were not comparable with the previous findings: the distinctive case marking resulted in more agreement errors. Participants made more errors in the conditions with singular subject and plural object clitic attractors compared to their counterparts with subject modifier attractors. However, these results contain two important confounds. The first of them is that, again, the attractor category is not controlled due to the limitation of the language. The second one is the syntactic function of the attractor: while one set of conditions has objects as attractors, the other set of conditions has subject modifiers, which resides in the exact phrase as the subject head, unlike the objects.\footnote{We are aware that an amplified effect with attractors in the object position is surprising and cannot be explained via representational accounts.}

In a subsequent experiment with a sentence completion task, Franck, Soare, Frauenfelder, and Rizzi (2010) again tested the role of distinctive case marking. In this experiment, they only used objects as attractors, thus eliminating the structural confound that was present in \citeand{FranckEtAl2006}. They manipulated the number of the attractor (plural x singular) and the type of the attractor (postverbal object x preverbal object clitic). While the postverbal object forms were syncretic between the accusative and nominative marking, the preverbal object forms were distinctively accusative-marked. The participants in this experiment were given infinitival forms of the verbs (shown in small caps) and asked to complete the sentence by conjugating the verb correctly. One set of experimental conditions was provided in (\ref{ex:franck10}).

\ea \label{ex:franck10}
  \ea {Preverbal Object Clitic}\\*
    \gll La vache le(-s) {souivre}.\\
    the cow it(-\Pl) to\_follow\\
    \glt `The cow follow$_{infinitival}$ it/them.' 
  \ex {Postverbal Object}\\*
    \gll La {vache} {souivre} le {chien(-s)}.\\
    the cow to\_follow the dog(-\Pl)\\
    \glt `The cow follow$_{infinitival}$ the dog(s)' 
  \z
\z

Their results were comparable to their previous experiment \citep{FranckEtAl2006}. Participants made more agreement errors in the conditions with distinctively case-marked object clitics than the conditions with postverbal syncretic objects. However, their results again contained two confounds. Firstly, the category of the attractor was not controlled. They compared the pronouns with full noun phrases. Secondly, the position of the attractor is different in their conditions. One can argue that the post-verbal position in French might be strongly associated with not being an agreement controller since no noun phrase that follows the verb can influence the agreement on the verb in French. 

Another study that dealt with this question was \cites{Slioussar2018} study using Russian case ambiguity. The author conducted three experiments, a sentence-formation task,\footnote{A sentence formation task is different from a sentence completion task. In the sentence formation task, all parts of a sentence are provided to the participants, and they were expected to form a meaningful sentence. When there is an ungrammaticality, let's say due to the verb number, they were expected to correct it. On the other hand, in a sentence completion task, participants are only provided with a preamble and are expected to complete the sentence according to their liking.} a speeded acceptability judgment, and a self-paced reading study. The same materials and manipulation were used in all experiments. The author manipulated the number of the number marking on the head and attractor noun (plural x singular), the case of the attractor (accusative x genitive), and the verb number (plural x singular). 

Russian is a fusional language that does not make use of articles with definite nouns. Like German, Russian also has three genders: masculine, feminine, and neuter. Depending on the gender and number, specific case suffixes can have the same surface form and be ambiguous. For \cites{Slioussar2018} study, two such ambiguities are of importance: accusative-nominative and nominative-genitive ambiguity. 

Within the conditions in which the attractor is marked with the accusative case, all attractors were ambiguous between the accusative and the nominative marking. These conditions were similar to the experiments done in English: the attractor `cabinets' in the sentence `\emph{The key to the cabinets is rusty,}' is ambiguous between the accusative and the nominative marking. However, this ambiguity stays at the level of form and does not result in different syntactic structures

As for the conditions with a genitive marking on the attractor, all attractors were not ambiguous between the genitive and the nominative marking. While plural attractors were unambiguously marked with the genitive case, the forms of singular attractors with genitive cases were the same as if they were plural nouns with a nominative case. Table \ref{tab:russianAmb} shows the conditions and the ambiguities present in \cites{Slioussar2018} experiments. 


\begin{table}[hbt!]
  \caption{Ambiguities between Cases in Their Singular and Plural Form}
  \vspace{10pt}
  \begin{tabular}{rcc}
    \hline
              & Accusative     & Genitive \\ \hline
    Singular  & \Nom.\Sg{}     & \Nom.\Pl{}   \\
    Plural    & \Nom.\Pl{}     & No Ambiguity  \\ \hline
  \end{tabular}

  \label{tab:russianAmb}
\end{table}

By utilizing these ambiguities between cases given in Table \ref{tab:russianAmb}, the author tested the role of ambiguity with experimental conditions presented in (\ref{ex:sli18}).

\ea \label{ex:sli18}
  \ea {Accusative Attractors}\\*
    \gll {Cena/ceny} na {produkt/produkty} byla/byli nizkoj/nizkimi {iz za ploxogo ka\v{c}estva syr'ja.}\\
    price.\Nom.\Sg/\Pl{} on product.\Acc.\Sg/\Pl\textsubscript{$=$\Nom.\Sg/\Pl} was/were low.\Sg/\Pl{} {because of poor quality of raw materials}\\
    \glt `The price(-s) on the product(-s) was/were low because of the poor quality of raw materials.'
  \ex {Genitive Attractors}\\*
    \gll {Vyder\v{z}ka/vyder\v{z}ki} iz knigi/knig byla/byli kratkoj/kratkimi {dlja upro\v{s}\v{c}enija processa zapominanija.}\\
    conclusion.\Nom.\Sg/\Pl{} from defeat.\Gen.\Sg\textsubscript{$=$\Nom.\Pl}/\Pl\textsubscript{no ambiguity} was/were brief.\Sg/\Pl{} {to simplify the memorization process}\\
    \glt `The excerption(-s) from the book(-s) was/were brief to simplify the memorization process.'
  \z
\z

Their results were comparable with Dutch, English, and German findings and contradicted French findings. In plural heads, they found no effect of attractor number, case marking, or interaction. However, the picture was different with singular heads. In the production experiment, Participants made more agreement errors with conditions where the marking of the attractor was syncretic with plural nominative marking compared to non-syncretic (unambiguous) conditions. More importantly, singular genitive-marked attractors that are syncretic with plural nominative marking induced more agreement errors than the plural genitive-marked attractors which are not syncretic. 

Similar findings were also observed in comprehension studies. In the speeded acceptability judgment task, participants found ungrammatical sentences acceptable more often when the attractor was singular and marked with the genitive case compared to the condition where the attractor was plural and marked with the genitive case. Participants read the same conditions faster than other ungrammatical conditions in the self-paced reading experiment. 

% Possessive marking on the attractor?

Lastly, \citeand{AvetisyanEtAl:2020} conducted one sentence completion production experiment and two self-paced reading experiments to test agreement attraction effects in Eastern Armenian. In their first self-paced reading experiment (Experiment 2), they used non-intervening attractors as in (\ref{ex:av20}). They manipulated the number of the attractor and the number of the embedded verb. They wanted to confirm that number agreement attraction effets surfaces in Eastern Armenian.

\ea \label{ex:av20}
  \ea[*]{{Plural Attractor, Ungrammatical}\\*
    \gll {Nkari\v{c}-ner-\"e}, or-onc' {k'andakagor\c{c}-\"e} arhamarh-ec'-in c'owc'ahandesi \"ent'ac'k'owm, {vagowc' mekowsac'vel} {en arvestagetneri} \v{s}r\v{j}anakic'.\\
    painter-\Pl.\Nom-\Def{} that-\Pl.\Acc{} sculptor.\Sg.\Nom-\Def{} ignore-\Aor-\Tpl{} exhibition during {long been ostracized} {are artists'} circle.\\
    \glt `The painters that the sculptor ignored$_{PL}$ during the exhibition have long been ostracized from the art community.'}
  \ex[*]{{Singular Attractor, Ungrammatical}\\*
    \gll {Nkari\v{c}-\"e}, or-i-n {k'andakagor\c{c}-\"e} arhamarh-ec'-in \ldots{} \\
    painter.\Sg.\Nom-\Def{} that-\Sg.\Acc-\Def{} sculptor.\Sg.\Nom-\Def{} ignore-\Aor-\Tpl{} \ldots{} \\
    \glt `The painter that the sculptor ignored$_{PL}$ during the exhibition has long been ostracized from the art community.'}
  \ex[]{{Plural Attractor, Grammatical}\\*
    \gll {Nkari\v{c}-ner-\"e}, or-onc' {k'andakagor\c{c}-\"e} arhamarh-ec' \ldots{} \\
    painter-\Pl.\Nom-\Def{} that-\Pl.\Acc{} sculptor.\Sg.\Nom-\Def{} ignore-\Aor.\Tsg{} \ldots{} \\
    \glt `The painters that the sculptor ignored$_{SG}$ during the exhibition have long been ostracized from the art community.'}
  \ex[]{{Singular Attractor, Grammatical}\\*
    \gll {Nkari\v{c}-\"e}, or-i-n {k'andakagor\c{c}-\"e} arhamarh-ec' \ldots{} \\
    painter.\Sg.\Nom-\Def{} that-\Sg.\Acc-\Def{} sculptor.\Sg.\Nom-\Def{} ignore-\Aor.\Tsg{} \ldots{} \\
    \glt `The painter that the sculptor ignored$_{SG}$ during the exhibition has long been ostracized from the art community.'}
  \z
\z

Their results showed that participants read ungrammatical sentences with plural attractors faster than their singular counterparts. Moreover, ungrammatical sentences with plural attractors were read as fast as the grammatical conditions.

In their second self-paced experiment (Experiment 3), they included four more conditions as in (\ref{ex:av20more}), in which they have used an attractor with a mismatching case-marking with the head. In their previous experiment, both the attractor and the head noun had the same case: Nominative. In their new conditions, all attractors are marked with an accusative case and have surface form with an \emph{-in} ending.

\ea \label{ex:av20more}
  \ea[*]{{Plural Attractor, Ungrammatical}\\*
    \gll {Nkari\v{c}-ner-i-n}, or-onc' {k'andakagor\c{c}-\"e} arhamarh-ec'-in c'owc'ahandesi \"ent'ac'k'owm, {vagowc' mekowsac'rel} {en arvestagetneri} \v{s}r\v{j}anakic'.\\
    painter-\Pl-\Acc-\Def{} that-\Pl.\Acc{} sculptor.\Sg.\Nom-\Def{} ignore-\Aor-\Tpl{} exhibition during {long ostracized} {are artists'} circle.\\}
    \glt `They have long ostracized from the art community the painters$_{ACC}$ that the sculptor ignored$_{PL}$ during the exhibition.'
  \ex[*]{{Singular Attractor, Ungrammatical}\\*
    \gll {Nkari\v{c}-i-n}, or-i-n {k'andakagor\c{c}-\"e} arhamarh-ec'-in \ldots{} \\
    painter-\Sg.\Acc-\Def{} that-\Sg.\Acc-\Def{} sculptor.\Sg.\Nom-\Def{} ignore-\Aor-\Tpl{} \ldots{} \\
    \glt `They have long ostracized from the art community the painter$_{ACC}$ that the sculptor ignored$_{PL}$ during the exhibition.'}
  \ex[]{{Plural Attractor, Grammatical}\\*
    \gll {Nkari\v{c}-ner-i-n}, or-onc' {k'andakagor\c{c}-\"e} arhamarh-ec' \ldots{} \\
    painter-\Pl-\Acc-\Def{} that-\Pl.\Acc{} sculptor.\Sg.\Nom-\Def{} ignore-\Aor.\Tsg{} \ldots{} \\
    \glt `They have long ostracized from the art community the painters$_{ACC}$ that the sculptor ignored$_{SG}$ during the exhibition.'}
  \ex[]{{Singular Attractor, Grammatical}\\*
    \gll {Nkari\v{c}-i-n}, or-i-n {k'andakagor\c{c}-\"e} arhamarh-ec' \ldots{} \\
    painter.\Sg.\Acc-\Def{} that-\Sg.\Acc-\Def{} sculptor.\Sg.\Nom-\Def{} ignore-\Aor.\Tsg{} \ldots{} \\
    \glt `They have long ostracized from the art community the painter$_{ACC}$ that the sculptor ignored$_{SG}$ during the exhibition.'}
  \z
\z

Their results showed no evidence towards the hypothesis that case-matching attractors amplified agreement attraction effects. They found small speed-ups in ungrammatical conditions with case-matching but number mismatching attractors in post-critical regions that immediately follow the embedded verb. However, these facilitory effects were negligible due to their extremely small magnitude (-2ms CrI:[-24, 20]). 

All experiments in this chapter included case syncretism on the level of morphophonology. There were no additional possible readings in any of these experiments presented. When there was a syncretism between any two cases, this syncretism did not have a reflex in the syntax or the parsing of the sentence. In other words, there was no local ambiguity present: the syntactic function of the noun that exhibits case syncretism was evident at all times. 

In our investigation of case syncretism and local ambiguity, we are investigating a case where there are multiple likely parses when processing the attractor and the head noun. While the Marking and Morphing account does not have any inherent mechanism to incorporate local ambiguity into the agreement computation, the cue-based retrieval account would predict that agreement attraction would be affected depending on possible parses introduced. Since the attractor and the controller would have different different set of features in their chunks, we expect that participants would may use features from an erroneous parse that lingers even after the reanalysis on same cases.

The second commonality between these studies is that they only manipulated the case syncretism on the attractor and never on the head noun. Remember that the presence of notionally plural sentences only affected agreement attraction when head nouns were notionally plural. The manipulation on the attractor did not support the idea that notionally plural nouns might amplify the agreement effects. 

Drawing parallelism from the interaction of notional plurality and the agreement attraction case, manipulating the case syncretism on the head noun might furnish a clearer picture of the interaction between case syncretism and agreement attraction. This would also enable us an additional venue to investigate the differences between cue-based retrieval and the Marking \& Morphing accounts. While the cue-based retrieval account would expect no difference between manipulating the case-matching on the attractor or the head noun, the Marking \& Morphing account would predict a visible difference between the syncretic cases on the attractor and the head noun. However, one must note that even though the Marking \& Morphing account differentiates the role of the attractor and the head noun, there is no clear way to integrate case information in the spreading activation formula. One has to assume that the more evident the case information is, the more easily number of a subject phrase would be detected. 


To sum up, the findings on the interaction case-syncretism and agreement attraction are not clear and the puzzle is missing some essential parts. While some researchers find distinctive case marking to reduce the agreement attraction effects in languages like Dutch, Russian, English, and German, other researchers showed that distinctive case marking increases the magnitude of agreement attraction in French. More recently, distinctive case marking was shown to have a negligibly small effect on agreement attraction in Eastern Armenian. However, apart from Eastern Armenian (no effect), German (positive effect), and Russian (positive effect) experiments, all previous studies included important confounds that might have affected the results. In addition to this conflicting results, all case syncretism manipulations are form-related syncretisms and not syntactic ambiguities. Lastly, the studies in the number agreement attraction literature only manipulated the case syncretism on the attractor. Thus, the effect of the case-syncretism question still stays unanswered and underexplored since the data show conflicting results and certain elements which are shown to be of importance in the literature such as the syntactic disparity between the attractor and the controller are not tested. 





\section{Role of shallow processing in agreement attraction} \label{sec:shallow}

Recent studies in psycholinguistics presented a great deal of evidence that interpretations formed by the participants do not always reflect the linguistic input that they were provided \citep{EricksonMattson81,BartonSanford93,Ferreira2003,Christianson2016}. One study conducted by Christianson, Hollingworth, Halliwell, and Ferreira (2001 showed that after readings sentences like (\ref{ex:gpath}), participants gave a surprisingly high number of yes responses to both questions presented in (\ref{ex:gpath_q1}) and (\ref{ex:gpath_q2}).

\ea \label{ex:gpath} While Anna dressed the baby that was cute and cuddly played in the crib.
\ex 
    \ea \label{ex:gpath_q1} Did Anna dress the baby?
    \ex \label{ex:gpath_q2} Did the baby play in the crib?
    \z
\z

If the sentence were processed fully, we would expect yes responses only after the question in (\ref{ex:gpath_q2}) and only \emph{no} responses after the question in (\ref{ex:gpath_q1}). Their findings support the idea that participants may sometimes analyze the sentence partially. These findings also support the Good Enough approach to processing: participants do not form perfect representations of the sentence; instead, they construct a representation that is good enough for the task at hand \citep{ChristiansonEtAl2001}. 

In this thesis, what we refer to with shallow processing is close to the assumption of the Good Enough approach. We argue that participants, instead of processing the sentence in detail, may sometimes use other heuristics to complete the task in the experiment. Heuristics in decision-making has been studied previously (Kahneman, Slovic, Slovic, \& Tversky, 1982; Gigerenzer \& Selten, 2022). Some heuristics may include word order information \citep{TownsendBever2001}, animacy of nouns \citep{Lamers2007}, or plausibility of a sentence (Van Herten, Chwilla, \& Kolk, 2006). 

Another possibility is that participants may use form-related heuristics. Previous research has found and replicated the effects of phonological similarities in working memory and reading tasks \citep{CopelandRadvansky2001} and single-word production studies (Baayen, Dijkstra, \& Schreuder, 1997; Schreuder \& Baayen, 1997; Rastle \& Davis, 2008). The idea that phonological similarities affect the sentence processing is also tested in the agreement attraction literature. For example, \citeand{BockEberhard93} tested whether singular attractors with an plural-like ending in English (pseudoplurals) might induce agreement attraction effects like overtly-plural marked attractors. They conducted a production study with a sentence-completion task and used experimental conditions as in (\ref{ex:bockeberhard93}).

\ea \label{ex:bockeberhard93}
    \ea {Pseudoplural Attractor} \\* The {player} on the {course} \ldots{}
    \ex {Singular Attractor} \\* The {player} on the {court} \ldots{}
    \ex {Plural Attractor} \\* The {player} on the {courts} \ldots{}
    \z
\z

Considering previous findings on erroneous tense marking with verbs that end with /s/ and /z/ \citep{SM86}, they argue that agreement attraction may also be an inhibitory mechanism where participants opt-out repeating the same phonological elements of the plural markings (/s/ or /z/) in plural attractor conditions as in `\emph{The king of the island/z/ rule/z/.}' If that were the case, words like `\emph{course}' that ends with a /s/ sound would elicit agreement attraction effects comparable to sentences with a proper plural attractor. 

However, their results did not support the hypothesis that endings that are phonologically like plural would interfere with the subject-verb agreement. The rate of agreement errors was not comparable to the conditions with a plural attractor.

However, \citeand{HaskellMacDonald2003} showed that irregular plural nouns, which do not end with a canonical plural ending /s/ or /z/, induce reduced agreement attraction effects compared to regular plural nouns in their Experiment 3. However, the effect was only limited to cases when the head noun is collective and was not present in Experiment 2, where they used non-collective heads. In both experiments, they have manipulated the type of attractor (regular x irregular). Attractors were always plural, and head nouns were always singular. One set of experimental conditions for Experiments 2 and 3 is shown in (\ref{ex:hask032}) and (\ref{ex:hask033}), respectively. 

\ea \label{ex:hask032} {Non-collective Heads}
    \ea {Irregular Plurals} \\* The {room} for the sick {children} \ldots{}
    \ex {Regular Plurals} \\* The {room} for the sick {kids} \ldots{}
    \z
\ex \label{ex:hask033} {Collective Heads}
    \ea {Irregular Plurals} \\* The {class} of {children} \ldots{}
    \ex {Regular Plurals} \\* The {class} of {kids} \ldots{}
    \z
\z

When the head noun was non-collective, the participants made more agreement errors with regular plural. However, the difference between the conditions was not substantially different. With collective heads, even though participants occasionally completed both type of sentences with a plural agreement, the rate of erroneous agreement marker was substantially low with irregular plurals. Their findings suggested that overt plural marking may increase the probability of having agreement errors in certain conditions.  

However, the frequency effect should be taken into account when irregular plurals are tested due to the suggested interaction between irregularity and frequency effects (Allen, Badecket, \& Osterhout, 2003). To circumvent this problem, Brehm, Hussey, and Christianson (2020) conducted a self-paced reading experiment where they controlled the attractor's frequency and irregularity. They have manipulated the number of the attractor (plural x singular), the orthographical type of the attractor (atypical plural x typical plural), the frequency of the attractor (high x medium x low), and the number marking on the verb (plural x singular). One set of experimental conditions can be found in (\ref{ex:brehm}).

\ea \label{ex:brehm}
    \ea {Atypical, High Frequency}\\* The {physician} who cured the {man/men} occasionally was/were incorrect about the diagnosis.
    \ex {Typical, High Frenquency}\\* The {physician} who cured the {boy/boys} occasionally was/were incorrect about the diagnosis.
    \ex {Atypical, Medium Frequency}\\* The {celebrity} who promoted the {dress/dresses} seldom was/were seen without a big entourage.
    \ex {Typical, Medium Frequency}\\* The {celebrity} who promoted the {skirt/skirts} seldom was/were seen without a big entourage.
    \ex {Atypical, Low Frequency}\\* The {landscaper} who planted the {cactus/cacti} already was/were anticipating the dry summer.
    \ex {Typical, Low Frequency}\\* The {landscaper} who planted the {yucca/yuccas} already was/were anticipating the dry summer.
    \z
\z

They found that participants read verb-spillover regions (incorrect, seen, or anticipating) in ungrammatical sentences overall faster when there is a plural attractor than singular attractor counterparts. They also found a slow-down in the same regions with low-frequency attractors. However, they could not find any effect concerning their morpho-orthographical manipulation. Even though previous research on isolated words suggested a possible effect of morpho-orthography, they could not find any effect of spurious decomposition of final /s/ sound. Their findings align with the previous results on morpho-phonology of English plurals in agreement attraction.

However, all these studies were conducted in English, a language in which the concept of pseudoplural is not straightforward. From the perspective of morpho-phonology, the word \emph{course} is a pseudoplural since that the last sound in this word corresponds to the phonological output of plural marking. However, the same word may not be considered a pseudoplural from the perspective of morpho-orthography since the word ends with a vowel \emph{e}.

Turkish does not exhibit such discrepancy between morpho-phonology and morpho-orthography concerning plural marking: All \emph{-lar} or \emph{-ler}'endings are pronounced the same way. However, apart from certain loan words like \emph{dolar} and \emph{ekler}, meaning \emph{dollar} and \emph{eclair}, respectively, pseudoplurals are extremely rare in Turkish. Thus, we could not test the effect of shallow processing and form heuristics using pseudoplurals. On the other hand, Turkish uses the same morpheme (\emph{-lAr}) for marking the plural agreement on verbs and plurality on nouns. We utilized this feature of Turkish and tried to test the use of form-related heuristics in agreement attraction and to induce agreement attraction effects using form-wise identical, but feature-wise different \emph{-lAr} markings in Section \ref{ch:exp2}. 


\section{Role of bias in agreement attraction} \label{sec:bias_aa}

Psycholinguistics mainly deals with participants' judgments in the experimental environment using tasks including a yes-no question, self-paced reading, and Likert scales. One of the most central questions in this endeavour is whether we can assume that these experiments truly measure the acceptability of sentences provided. Signal Detection Theory, one of the theories that model participants' responses, argues that many factors, such as response bias, might affect the experimental results and participants' responses \citep{MacmillanCreelman2005}. Signal Detection Theory assumes that even the categorical responses like yes and no are actually continuous, and participants categorize them according to decision criteria. 

The first application of Signal Detection Theory to acceptability judgments was made by \citeand{BaderHaussler2010}. Following \citeand{GreenSwets1966} and \citeand{MacmillanCreelman2005}, they argue that the judgment process is two-fold. Participants first compute a continuous value of acceptability for the sentence they were prompted to read. Then, they choose the category to which this continuous value belongs. Their results showed a strong correlation between the continuous magnitude estimation and the categorical yes-no responses. One interesting question is how the decision criteria can be determined in experiments with no accompanying data like magnitude estimation and whether or not there are underlying phenomena that might change the decision criteria depending on the study and the participants. One such possible underlying factor that affects the experimental results is response bias.

Response bias is participants' tendency to choose an option over another possible option with no necessary evidence towards any options \citep{MacmillanCreelman2005}. As Rotello, Heit, and Dub{\'e} (2015) presented, response bias might induce such experimental results that they could be mistaken for an effect on the percentage of correct responses. They also showed that increasing the power of the experiment by conducting the experiment with a bigger participant pool or more trials per subject worsened the problem of response bias even more. One way to overcome this problem is to integrate the bias value into the analysis of the experimental results. 

To our knowledge, there is only one experiment that introduced the response bias manipulation to the agreement attraction phenomenon: \citeand{HammerlyEtAl2019}. They assumed that the plural attractor's lack of interference in ungrammatical sentences was due to the participants' a priori bias to give more yes responses. Via three speeded acceptability judgment experiments, they showed that when participants' response bias is manipulated using instructions and the ratio of ungrammatical to grammatical sentences in an experiment, the agreement attraction patterns in the percentage of acceptable responses also change. They have manipulated the number of the attractor and the verb (plural x singular) in all of their experiments. Within experiments, they manipulated the instruction and the ratio of ungrammatical sentences. Their first experiment did not use any special instructions and used an equal number of ungrammatical and grammatical sentences. In their second experiment, participants were informed that 2/3 of the sentences they would see in the experiment would be ungrammatical. They also modified the ratio of ungrammatical sentences in the experiment such that \%64 of the overall items were ungrammatical. In their third experiment, participants were told most sentences in the experiment were ungrammatical. They used the same ratio of ungrammatical sentences in their third experiment. One set of experimental items can be found in (\ref{ex:ham_ch2}).

\ea \label{ex:ham_ch2}
  \ea[*]{\label{ex:ham_ch2plpl}{Plural Attractor \& Ungrammatical (Plural Verb)} \\* The {friend} of the {nurses} frequently visit.}
  \ex[]{{\label{ex:ham_ch2plsg}Plural Attractor \& Grammatical (Singular Verb)} \\* The {friend} of the {nurses} frequently visits.}
  \ex[*]{{\label{ex:ham_ch2sgpl}Singular Attractor \& Ungrammatical (Plural Verb)} \\* The {friend} of the {nurse} frequently visit.}
  \ex[]{{\label{ex:ham_ch2sgsg}Singular Attractor \& Grammatical (Singular Verb)} \\* The {friend} of the {nurse} frequently visits.}
  \z
\z

Their results showed a clear effect of response bias on the interference of plural in grammatical sentences. Their first experiment with no bias manipulation showed mainstream agreement attraction effects: no effect of number marking in grammatical conditions and an apparent impact of number marking in ungrammatical conditions on yes responses. Participants accepted sentences like (\ref{ex:ham_ch2plpl}) more often compared to (\ref{ex:ham_ch2sgpl}). This interaction was reduced as the participants' response bias toward yes responses was reduced. In Experiment 3, they found that participants almost made as many errors in grammatical sentences with plural attractors (\ref{ex:ham_ch2plsg}) as they did with ungrammatical sentences with plural attractors (\ref{ex:ham_ch2plpl}).

Chapter \ref{ch:exp3} attempts to replicate and clarify the findings on response bias and agreement attraction by \citeand{HammerlyEtAl2019}. To this end, we conducted a speeded acceptability judgment task in Turkish using another syntactic construction: a complex noun phrase with a genitive-marked modifier. Moreover, we only used filler items in our response bias calculation to have a clearer picture of response bias.


