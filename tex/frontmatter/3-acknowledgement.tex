\chapter*{\MakeUppercase{acknowledgements}}

Putting my name on this thesis feels like a fraud. Every detail within this thesis has been inspired and enriched by people in Bo\u{g}azi\c{c}i University, first of which is Pavel Loga\v{c}ev. He taught me everything I know about psycholinguistics, statistics, experimental design, and dealing with hardship and life during graduate studies. His questions and his support to not believe in the results pushed me (and many students in Bo\u{g}azi\c{c}i) to be better researchers. At first, we thought that he is just doing it to torture us and have fun, but at some point we all realized he changed how we looked at scientific work and linguistics. Throughout the thesis, I used the first-person plural pronoun, and not the singular one. The main reason is that not a single sentence of this thesis would be possible without his help.

He was always there when I needed emotional support. He was the first person I called when I learned my father passed away. Whenever I felt down and needed a friend to talk to he supported me and listened me whenever he is available. It feels like finding a friend and a mentor like him is impossible, and I desperately hope this thesis is a good-enough acknowledgement of my appreciation to his friendship and mentorship. 

Balk{\i}z \"Ozt\"urk is the role model educator and researcher that everyone would aspire to be like one day. Without her help and her encouraging, I would not be able to achieve the half of my presentations or publications during my MA. She introduced me to nanosyntax and the Universal Dependencies. She was willing to read, edit, and comment on every work I did. She always fought for my rights as a student, researcher, and teaching assistant and always reminded me to stood up for myself. She encouraged me and everyone in the department to believe in ourselves, submit our work and present it whenever possible.

I also wish to express my gratitude to members of my committee: \"Umit Atlamaz and Serkan \c{S}ener. \"Umit Atlamaz was always there to answer my questions on syntax and to ease my worries about Ph.D. applications. He is unique in the sense that he seems very chill and easy-going, and at the same time very serious about his work. He is extremely friendly and would go under any burden to help others. Not only that, but he also prepares the best coffee in the campus except for me, of course. I am also thankful to Serkan \c{S}ener. We first met when he was a committee member of a thesis defense in Bo\u{g}azi\c{c}i. He is extremely easy to talk to and have a very good understanding of theoretical and experimental work. He was the first person to encourage me to pursue my broad interests. It was also very nice of him to accept giving an invited talk in one of the conferences we organized. I still remember him encouraging me with his head nods in my first conference talk.

During my MA, I spent the most time in the assistant room. I was very lucky to have so many fun, great and exceptionally smart people as a colleague. Hande Sevgi and Duygu G\"oksu was my official `bac{\i}lar{\i}m.' They taught me every little detail about TAing and administrative issues in the department. We would start our day with a breakfast at the back of the department, work hard during the day, and have fun after 5 pm. I remember going to Bebek, dancing at a rave, watching movies in the department, celebrating end of the semester with them. When we were talking about applying to Ph.D., I always dreamed of going to New England area to hang out with them. Furkan Dikmen joined us a bit later. He was a shy person at the beginning, but I was able to break his tough shield with my humor. I am glad that I was able to reach him since without him my MA years would be extremely boring. We were able to find something to laugh with no problem or some linguistic phenomenon to discuss to the depth. It is a pity that we never get to sit down and wrote something together. At last, Muhammed {\.I}leri joined us. It was immediately before the COVID hit Turkey, so we never got to share an office space for too long until my last year. I sometimes thought he got bored with my and Furkan's child-like behavior. But, he made sure that was not the case.

I very much appreciate the help and advice from my professors in my department. Having impromptu talks with Sumru \"Ozsoy, trying to clever ways to answer intriguing questions of Elena Guerzoni, before class discussions on Turkish morphophonology with Asli G\"oksel, hallway chats on life in linguistics with Didar Akar, and most importantly daily sanity-checks with Mine Nakipo\u{g}lu where we first deploringly talked about what was going on around us and then thanked linguistics and the work done for giving us something to be excited for were what the things that kept me sane in my first years. The project we did with Stefano Canalis, realizing the mind-boggling nature and richness of minority languages in Asia Minor with Metin Ba\u{g}r{\i}a\c{c}{\i}k, the joy of teaching Turkish in summers with Kadir G\"okg\"oz and Ceyda Arslan-Kechriotis, and her life-giving and morale-boosting pep-talks made me even more convinced that I was lucky to be in this department and in this field. But without \"Omer Demirok and Deniz \"Ozy{\i}ld{\i}z, I do not know how I would not die out of anxiety over the smallest things. They were always one message away to explain me a very complex idea, to soothe me when I get too stressed to work, or to encourage me to follow even the stupidest idea that I had. Whenever I have a question on semantics, morphology, or living in the States, I first go to our WhatsApp or Messenger chats and check the messages before searching online. 

I would also like to extend my thanks to Pavel Caha and Michael Starke. Throughout my MA, Pavel met me every week online to discuss any material I choose, put a lot of effort so that I could visit Masaryk University, and made sure that I felt welcomed at Brno and at any nanosyntax meeting. 

I was also lucky to have a lot of friends to thank during my stay in Bo\u{g}azi\c{c}i. Their presence created a really nice environment to flourish great ideas and have fun at the same time. Alper Ahmeto\u{g}lu and \c{C}a\u{g}la Aksoy beared with me even though I was the lamest friend they had. One of the greatest moments of my life was in Adana with them. G\"urkan \.Izmirligil and Sercan Demiralp are the ones who never leave me alone in my ten-year-long BA and MA adventure in Bo\u{g}azi\c{c}i. The Manzara and South Campus will always remind me the times we had fun in Oyun Kulubu. Ka\u{g}an K\"u\c{c}\"ukco\c{s}kun and Atakan Kaya were there for me in my worst time and helped me get it together so that I would write my thesis.

In the linguistic department, I was lucky to have friends like Furkan Atmaca, Duygu \c{C}ak{\i}r, Dilan Tanal, B\"u\c{s}ra Mar\c{s}an, Noyan Dokudan, Asl{\i} Kuzgun, Burak \c{C}avu\c{s}o\u{g}lu, Aref Milani, Assem Amirzhanova, Sercan Karaka\c{s}, \"Umit Tun\c{c}er, and Ege Baran Dalmaz. All of them touched my life and the work I have done in many ways. They inspired me to work hard, supported me in every possible way that they could and encourage me to be a better version of myself. They taught me not to give in the toxic culture of academia and tidied up my sharp edges. I would like to thank Furkan Atmaca specifically since without his help in R and \LaTeX{} this thesis would be so hard to complete. He is also the architect of this \LaTeX{} template. He was extremely fun to hang around and helped me entertain many crazy ideas. B\"u\c{s}ra is one of a kind person and can simply bring joy by her sheer existence. I will always remember the FilmEkimi opening with Duygu and the horrendous movie `Titane.'

One of the greatest joy during my MA was teaching. Our department allowed me to TA more than 25 classes, which taught me a lot about working in academia and articulating my knowledge better. I will carry this experience in my CV as an honor badge as I cannot any other place to head-start teaching and academia. 

I am also very lucky to meet Merve Aydar in my last year here in Turkey. She showed me all the good things that there is to know about Istanbul. Her presence transformed my life, my painfully long thesis writing process into a bliss. I frequently found myself feeling lucky to have her support during this process. She is a gem, and I am very fortunate to know her. 

I dedicate this thesis to my dad, my mom, and my sister. My mom is the pinnacle of patience. She endured more than any human can endure. She supported me in every decision I took even if it meant not seeing me for a long time. I am sorry mom, I did not call you enough. And my beautiful sister. Without her funny messages and photos of our beautiful dog Hera, my life would be extremely dull. I always looked forward to seeing her in my holidays. My dad, who passed away in 2019, showed me what hard work could accomplish and dedicated his last years to ensure that I was never in pain or in need. From the early days of my high school, he, a polyglot, insisted that I was doing something related to languages, computer science, and psychology. I hope you are proud of me, dad. 
