\chapter*{ABSTRACT\\ \ttitle}
\pagenumbering{roman}
\setcounter{page}{4}

In this thesis, I investigate the existing agreement attraction effects in Turkish and how these effects interact with various phenomenon such as (i) case syncretism and local ambiguity, (ii) form heuristics, (iii) response bias, and (iv) honorific readings. Previous studies have shown that speakers occasionally find ungrammatical sentences violating number agreement acceptable when there is another noun sharing same number with the verb, in other words exhibited agreement attraction. \citet{LagoEtAl2019} found that genitive-possessive structures were able to induce agreement attraction effects within native Turkish speakers in a speeded acceptability experiment. However, due to the nature of the Turkish and acceptability studies, there are multiple alternative explanations for the existing effects. This thesis aims to weed out possible confounds and clarify the effects by conducting four speeded acceptability judgment experiments. We showed (i) that case-ambiguity on the head noun does not play a role in Turkish agreement attraction (Experiment 1, $N=118$), (ii) that participants do not use form-driven-processing-strategies to answer judgment questions (Experiments 2A, $N=80$, and 2B, $N=95$), (iii) that response bias induced ungrammaticality illusion and only decreased the magnitude of grammaticality illusion (Experiment 3, $N=114$), and (iv) that a possible honorific reading does not license superfluous plural marking at the verb (Experiment 4, $N=174$). Together, our results challenge cue-based retrieval accounts of agreement attraction and can be accommodated by accounts that assume attraction occurs due to erroneous encodings.

\newpage

%\mkbibparens{\citeauthor{Greenberg}\nameyeardelim \citeyear{Greenberg}\postnotedelim as cited by \citeauthor{Bagriacik2020} \mkbibparens{\citeyear{Bagriacik2020}}}

\chapter*{\"{O}ZET\\ \ttitletr}

Bu tezde T\"urk\c{c}ede daha \"once bulgulanm{\i}\c{s} uyum benze\c{s}mesi ve bu bulgular{\i}n (i) durum ayn{\i}la\c{s}mas{\i} ve yerel belirsizlik, (ii) bi\c{c}im temelli sezgisel stratejiler, (iii) tepki yanl{\i}l{\i}\u{g}{\i} ve (iv) olas{\i} sayg{\i}l{\i} dil kullan{\i}m{\i} okumas{\i} gibi olgularla etkile\c{s}imi incelenmektedir. \"Onceki \c{c}al{\i}\c{s}malar g\"ostermi\c{s}tir ki konu\c{s}ucular, t\"umce i\c{c}inde y\"uklem ile ayn{\i} say{\i} \c{c}ekimini payla\c{s}an ba\c{s}ka bir ad \"obe\u{g}i bulundu\u{g}u vakit, say{\i} uyumunu ihlal eden t\"umceleri s{\i}k s{\i}k kabul edilebilir bulmu\c{s}lar, di\u{g}er bir deyi\c{s}le uyum benze\c{s}mesi etkileri g\"ostermi\c{s}lerdir. Lago v.d. (2019) iyelik \"obe\u{g}i yap{\i}lar{\i}n{\i}n kullan{\i}ld{\i}\u{g}{\i} deneylerde anadili T\"urk\c{c}e olan konu\c{s}ucular{\i}n sabit-h{\i}zl{\i} dilbilgisel yanl{\i}l{\i}k de\u{g}erlendirmelerinde uyum benze\c{s}mesi ger\c{c}ekle\c{s}tirdi\u{g}ini bulgulam{\i}\c{s}t{\i}r. Fakat, T\"urk\c{c}enin ve dilbilgisel yanl{\i}l{\i}k \c{c}al{\i}\c{s}malar{\i}n{\i}n do\u{g}as{\i}ndan gere\u{g}i baz{\i} alternatif hipotezler geli\c{s}tirilebilir. Bu tez d\"ort sabit-h{\i}zl{\i} dilbilgisel yanl{\i}l{\i}k de\u{g}erlendirme deneyi kullanarak bu olas{\i} hipotezleri, di\u{g}er bir deyi\c{s}le parazit fakt\"orleri, elemek ve etkileri netle\c{s}tirmeyi ama\c{c}lamaktad{\i}r. Yapt{\i}\u{g}{\i}m{\i}z deneylerle (i) ba\c{s} \"ogede durum ayn{\i}la\c{s}mas{\i}n{\i}n T\"urk\c{c}edeki uyum benze\c{s}mesinde rol oynamad{\i}\u{g}{\i}n{\i} (Deney 1, $N=118$), (ii) kat{\i}l{\i}mc{\i}lar{\i}n dilbilgisel yanl{\i}l{\i}k sorular{\i}n{\i} cevaplarken bi\c{c}im-g\"ud\"uml\"u-i\c{s}leme-stratejisi kullanmad{\i}\u{g}{\i}n{\i} (Deney 2A, $N=80$, ve 2B, $N=95$), (iii) tepki yanl{\i}l{\i}\u{g}{\i}n{\i}n dilbilgisid{\i}\c{s}{\i}l{\i}k yan{\i}lsamas{\i}na sebebiyet verdi\u{g}ini ve dilbilgisellik yan{\i}lsamas{\i}n{\i} azaltt{\i}\u{g}{\i}n{\i} (Deney 3, $N=114$), son olarak da (iv) olas{\i} bir sayg{\i}l{\i} dil kullan{\i}m{\i} okumas{\i}n{\i}n y\"uklemdeki fazla \c{c}o\u{g}ul eki kullan{\i}m{\i}n{\i} yetkilendirmedi\u{g}ini (Deney 4, $N=174$) g\"osterdik. Birlikte ele al{\i}nd{\i}\u{g}{\i}nda, sonu\c{c}lar{\i}m{\i}z ipucu-odakl{\i} geriye getirme izahatlerine meydan okumakta olup benze\c{s}menin hatal{\i} kodlama dolay{\i}s{\i}yla ger\c{c}ekle\c{s}ti\u{g}ini varsayan izahatlerle a\c{c}{\i}klanabilmektedir.