
\chapter{DISCUSSION} \label{ch:discussion}

This thesis set out to understand the processing errors in the subject-verb dependencies in Turkish. The focus has been to clarify the agreement attraction effects in Turkish and eliminate possible confounds in the existing literature. We wanted to understand what aspects of agreement attraction findings could be explained with extra-linguistic phenomena like storing erroneous parses, using form heuristics, or having a response biases. To this end, we determined three possible confounds to test:

\begin{enumerate}[label=(\roman*)]
    \item {A lingering effect of an erroneous parse due to case syncretism:} Local ambiguity due to a case syncretism between a subject compatible marking and a non-compatible marking on the subject head may lead participants to retrieve the attractor as an agreement controller.
    \item {A task-specific response strategy using form heuristics:} Unlike other languages, Turkish plural marker on nouns and the plural agreement marker on verbs are homophones. Assuming participants engage in shallow processing, they may form a strategy where they answer questions by matching the final plural agreement with a previous plural marking in the sentence.
    \item {Response bias as an underlying cause of existing effects:} Patterns of agreement attraction in yes percentages might be due to participants' a priori tendency to give yes responses. \citeand{HammerlyEtAl2019} showed the true nature of agreement attraction by getting rid of this existing response bias. Turkish agreement attraction might also be affected by the presence of an underlying response bias.
\end{enumerate}

To test these interactions between the aforementioned phenomena and agreement attraction, we conducted three speeded acceptability judgments. Section \ref{ch6predictions} puts forward the predictions of the cue-based retrieval and the Marking and Morphing acconts of agreement attractions. Section \ref{ch6summary} summarizes our findings in these experiments in broad strokes. Sections \ref{ch6case}, \ref{ch6form}, and \ref{ch6bias} will discuss the implications of these experiments. In Section \ref{ch6syntax}, we firstly discuss how two syntactic theories of sentence object relative clauses predict contrasting patterns in the Marking and Morphing account of attraction. We then, discuss a possible confound we have not covered in this thesis: the possibility of honorific reading in Turkish agreement attraction effects in Section \ref{ch6hon}.


\section{Predictions} \label{ch6predictions}

Before summarizing our findings, we would like to lay out the predictions of the two most important accounts of agreement attraction.

\subsection{Experiment 1}
In our first experiment, we manipulated the overt-case marking of the subject head. 

\subsubsection{Cue-based retrieval account}

In the cue-based retrieval account, the comprehension process in language utilizes a structured search to satisfy dependencies. Certain elements, such as verbs, trigger a search by providing specific cues, such as [+{pl}]. The dependency is satisfied when there is a match between the cues provided by the verbs and the features from the previous chunks. Attraction occurs when multiple chunks are considered for possible retrieval. 

The exact information stored in the chunks and the same cues utilized in this process is still an open debate \citep{ArnettWagers:2017}. Morphological realization of the abstract case may also be another feature stored in the chunk and be used as a cue in the retrieval process, as recent studies on Russian indicates \citep{SlioussarMalko2016, Slioussar2018}.

Thus, we believe that the cue-based retrieval account would expect reduced attraction effects when the case on the subject head is not ambiguous. 

\subsubsection{Marking and morphing account}

In the Marking and Morphing account, the attraction occurs due to the probabilistic spread of the plurality from the attractor to the root node of the subject. This spread activation depends on the number information of every phrase within the subject and their syntactic distance to the root node. 

Due to the nature of the activation spread, the Marking and Morphing account would not predict any additional interference due to the case-related manipulation. 

\subsection{Experiments 2A \& 2B}
In experiments 2A and 2B, we tested the interaction between the form-advantage and agreement attraction. We utilized the homophony between nominal plural marking and verbal plural agreement in Turkish. 

\subsubsection{Cue-based retrieval account}
We argued that if participants utilize form-related features rather than abstract features, we should also observe attraction effects with verbal attractors. 

Many implementations of the cue-based retrieval account do not specify any cue for the category of the agreement controller or the category of the stem of the agreement controller. Since Turkish reduced object relative clauses may also serve as agreement controllers, we believe the cue-based retrieval account would predict a presence of an attraction effect in ungrammatical sentences. 

\subsubsection{Marking and morphing account}

Following the intricacies of the spreading activation formula, we believe that the Marking and Morphing would also expect a presence of an attraction effect in ungrammatical sentences. However, since the syntactic position of the verbal plural agreement is embedded more deeply, we expected either significantly decreased attraction effects or a lack of its presence due to our experimental choices.   

\subsection{Experiment 3}
In experiment 3, we tested whether manipulating a priori response bias impacts the effect of attractor number in grammatical sentences. 

\subsubsection{Cue-based retrieval account}
Under the assumptions of the cue-based retrieval account, changing a priori bias would not result in an amplified effect of attractor number in grammatical sentences. This lack of increase is because attraction is expected to surface only in cases with more than a single match. In grammatical sentences, the dependency between the verb and the subject is satisfied without any problem. The number and the subjecthood features, such as case and syntactic position, fully match the cues provided by the verb in grammatical sentences. Thus, no other candidate for the agreement controller role should be entertained.


This understanding of attraction effects necessitates that the grammaticality asymmetry is due to the characteristics of sentence processing. Given the assumptions of this model, we believe that the cue-based retrieval account would expect no attraction effects when the a priori response bias is manipulated. 

\subsubsection{Marking and morphing account}

A priori response bias of the participants is not implemented directly in the Marking and Morphing account. As far as we know, this account is not equipped to integrate response bias into the attraction phenomenon. 

However, unlike the cue-based retrieval account, The Marking and Morphing account does not expect a grammaticality asymmetry: the effect of plural attractor should be comparable in grammatical and ungrammatical sentences. While this asymmetry is a direct result of the sentence processing mechanisms in the cue-based retrieval phenomenon and is intertwined with the attraction process, the Marking and Morphing account is agnostic to this phenomenon.  

A recent study by \citeand{HammerlyEtAl2019} showed that this asymmetry is related to the nature of linguistic experimenting. They argue that participants have a general tendency to answer yes more often than no. They utilized \cites{Ratcliff1978} Drift Diffusion Model and showed that when the extra-linguistic factor bias is controlled, the predictions of the Marking and Morphing account hold.

We reasoned that \cites{HammerlyEtAl2019} data, manipulation, and findings should be replicated, given that the Drift Diffusion model account of decision making is not limited to a particular language, a particular structure, or a particular demographic. That is, participants with no a priori bias towards yes responses should also exhibit attraction effects in grammatical sentences.


\section{Summary of findings} \label{ch6summary}
Experiment 1 was concerned with the possible confound in \cites{LagoEtAl2019} study. They argued that Turkish native speakers accepted ungrammatical sentences with plural attractors more often than their singular attractor counterparts because the genitive case marking is usually used as a subject marker in Turkish. However, all sentences in their experiment had two possible parses until they encounter the matrix verb, which was the last element in the sentence. In one possible parse, participants formed a representation where the subject was a complex NP with a genitive-marked modifier. In the second possible parse, their representation included an embedded sentence with a genitive-marked subject and an accusative-marked object. We argued that the present agreement attraction effects might be due to this local ambiguity and lingering effects of not-completely abandoned parses. We disambiguated the subjects they used and aimed to replicate their findings. If the present effects were due to linguistic features, such as the [+{subj}] feature and did not result from an erroneous parse, we expected to find comparable results to \cites{LagoEtAl2019} findings. Given our data, our results contradicted our hypothesis and verified that case syncretism does not play a role in Turkish agreement attraction. 

Experiments 2A and 2B dealt with another possible hypothesis that might explain \cites{LagoEtAl2019} findings. Due to the unique feature of Turkish agreement attraction, we hypothesized that participants might use form-driven processing strategies, assuming that they engage in shallow processing. Unlike other languages in which agreement attraction is tested, Turkish nominal and verbal plural markings are homophonous. One possible explanation of Turkish agreement attraction findings is that participants do not fully process sentences and match two \emph{-lar} markings in a sentence to judge the grammaticality of sentences when they do not have sufficient information. On some occasions where they could recall that there was a plural present but could not recollect the exact host of the plural marking, they might deem sentences acceptable. To test this hypothesis, we used plural marked verbs of reduced relative clauses as attractors and expected comparable effects of plural marking in ungrammatical sentences in Experiment 2A. However, our results contradicted our hypothesis: participants were highly successful in detecting ungrammatically in RC attractor conditions independent of the presence of a plural attractor. 

In Experiment 2B, we included four new conditions to test whether our findings in Experiment 2A were because that participants have no way of associating the previous plural marking with grammaticality. Since Turkish plural marking on the verb is not obligatory, participants may not have a priori tendency to match two \emph{-lar} markers in sentences. For priming participants to consider our hypothesized matching mechanism, we included new conditions in which we had a complex NP with a genitive-marked modifier like the ones we used in Experiment 1. With new conditions, we expected our participants to accept ungrammatical sentences with a plural verbal attractor more often than their singular verbal attractor counterparts. We, again, found that participants did not make any additional judgment errors when there was a plural verbal attractor. However, we also found that the overall acceptability of ungrammatical sentences with genitive-marked attractors reduced substantially compared to Experiment 1. Even though we were not able to confirm that participants utilized form heuristics to complete the task, our results suggest that the task and the other conditions might influence the magnitude of the agreement attraction effects.

In Experiment 3, we tested whether the nature of the task might influence the mainstream patterns of attraction. With the nature of a task, we refer to the instructions and the number of ungrammatical and grammatical fillers. Recently, \citeand{HammerlyEtAl2019} found that participants made judgment errors in grammatical sentences almost as often as they did in ungrammatical sentences. Following \cites{Ratcliff1978} DDM model, they argued that participants had a priori response bias towards yes responses, and when this bias was neutralized through the instructions and the ratio of ungrammatical sentences to grammatical sentences, the main effect of plural attractor would be present independent of sentence grammaticality. Their results verified this hypothesis and supported attraction accounts based on representational errors rather \citep{EberhardEtAl2005} than retrieval errors \citeand{WagersEtAl:2009}. We wanted to replicate these findings in Turkish with a different syntactic structure since both grammaticality asymmetry, and DDM accounts are not limited to a single language, and their results were only attested in one language: English. When we assessed the response bias using fillers, we found that we could not manipulate participants' response bias. However, we also found that \citeand{HammerlyEtAl2019} also could not manipulate response bias according to our calculation of response bias using fillers. Thus, we grouped our participants into two using calculated bias estimates and not the experimental manipulation. Our results, using this grouping, confirmed theoretically significant aspects of \citeand{HammerlyEtAl2019}: With neutralized bias, participants judged grammatical sentences as ungrammatical when there was a plural attractor present. 

\section{Case syncretism} \label{ch6case}

As discussed in Chapters \ref{ch:accounts} and \ref{ch:exp1}, we wanted to check the effects of case syncretism in Turkish agreement attraction. Even though previous research on case syncretism presents a solid case for affecting sentence processing, the literature on the agreement attraction was not coherent. Experiments in initial studies mostly included confounds such as attractor type \citep[][in Dutch]{HartsuikerEtAl2003} and syntactic position \citep[][in French]{FranckEtAl2006}. Later, studies were conducted on other languages in which researchers could manipulate the case syncretism or distinctive case marking without introducing confounds. However, these results were also not conclusive: while Eastern Armenian did not show any interaction between case syncretism and agreement attraction \citep{AvetisyanEtAl:2020}, results from German and Russian experiments showed that when participants saw attractors with a case marking that is syncretic between two cases, they make more agreement errors with plural attractors in the vicinity than with plural attractors that carry distinctively marked case.

However, all these studies manipulated the case syncretism on the attractor. We can infer from \citeand{BockEberhard93} and \citeand{HaskellMacDonald2003} that the effect of a manipulation changes depending on whether the attractor or the head is affected by the manipulation. Notionally plural nouns do not appear to play a role in agreement attraction when they were first introduced in the attractor position \citep{BockEberhard93}. However, \citeand{HaskellMacDonald2003} showed that the notional plurality of nouns has a tremendous effect when it is introduced in the head noun.

Additionally, the previous experiments on case syncretism never introduced a local ambiguity. Even though the case on the attractor was syncretic, this case syncretism never resulted in syntactic ambiguity. The syntactic relation between the attractor and the subject head was clear. In \cites{LagoEtAl2019} Turkish experiment, however, participants were able to entertain two different syntactic structures until they saw the last element in every item. In one possible parse, the first DP, the attractor, could be interpreted as a genitive-marked modifier of the second NP, the subject head. In another possible parse, participants might entertain a syntactic structure involving an embedded sentence when they see a genitive-marked DP. The first DP could then be interpreted as the subject of an embedded sentence, while the second DP could be interpreted as a direct object of the embedded sentence.  

Drawing parallelism from the line of work in the notional plurality issue, a case syncretism between a non-subjecthood case and a subjecthood case might have played an important role in Turkish agreement attraction effects. Considering that the case syncretism introduces a local ambiguity, we hypothesized that the present agreement attraction effects in \cites{LagoEtAl2019} work might decrease or disappear when we disambiguated the case syncretism and used distinctively marked case on the head.

We saw that Turkish agreement attraction effects were not contingent on the local ambiguity and case syncretism. When we disambiguated the subject marking on the head, our results were comparable to \cites{LagoEtAl2019} findings. In both studies, there was an interaction between the presence of a plural attractor and grammaticality. Participants accepted ungrammatical sentences with plural attractors more often than the ones with singular attractors, and this effect was not present in grammatical sentences. 

Our results contradicted our hypothesis that lingering effects of an erroneous parse might affect the acceptability of the sentence. Either participants do not entertain the syntactic parse involving an embedded structure due to its being less economical to do so, or they quickly recover from the local ambiguity so that it does not affect the grammaticality judgment. Our experiment was not equipped to answer this question; however, a future study involving a self-paced reading experiment or an eye-tracking experiment might answer this question.

\section{Form heuristics} \label{ch6form}

Experiments 2A and 2B tested whether participants use form heuristics to complete the grammaticality judgment task. Chapters \ref{ch:accounts} and \ref{ch:exp2} discussed the potential reasons for us to entertain an alternative hypothesis for present agreement attraction effects in Turkish. Namely, participants may use a strategy based on matching \emph{-lar} morphemes when they could not judge the sentence reliably due to memory uncertainty. 

We compared sentences containing a reduced relative clause with an overt plural marking to sentences containing a genitive marked subject modifier to test this hypothesis. If participants used the form of \emph{-lar} markings to answer grammaticality judgments, we expected comparable effects in ungrammatical conditions with verbal attractors as well.

However, our results suggested that participants do not use form heuristics in Turkish agreement attraction effects. We could not find an effect of plural attractors in ungrammatical sentences within verbal attractor conditions. This finding was verified with an additional experiment, where we included minimally different genitive marked subject modifier conditions to the experiment. We took the lack of an effect in both experiments to indicate that participants did not use forms as a cue in the processing of subject-verb dependency and that the part of speech tag of the attractor was important even when the attractor was a nominalized relative clause.

On the other hand, our results showed a reduced magnitude of agreement attraction in genitive marked subject modifier conditions when the experiment included verbal attractor conditions. Even if we could not find an evidence of a form-heuristics-based mechanism, we could interpret these findings as mild evidence of task effects. The presence of a set of clearly detectable grammatical and ungrammatical subject-verb dependency conditions (i.e., verbal attractor conditions) might have reduced overall errors in other conditions, which included a genitive marked subject modifiers as attractors. 

We also found a small positive effect of the presence of a plural attractor in grammatical sentences in Experiment 2A. This finding was unexpected given previous agreement attraction studies in which the presence of a plural attractor either affected the acceptability of grammatical sentences negatively or did not affect them. We believe that a plural marking on a reduced relative clause could induce an impersonal reading, whereas the lack of a plural marking would require a specific subject in the context \citep{Kornfilt:2011}. We believe that the positive effect of plural attractors in grammatical sentences might be due to this difference between interpretations. Unfortunately, our results in ungrammatical sentences might also be affected with this interpretation difference. The presence of a plural marker on the reduced relative clause might be too marked to go unnoticed since it might have an impersonal reading contribution. 

\section{Response bias} \label{ch6bias}

Experiment 3 re-examined \cites{HammerlyEtAl2019} hypothesis that the grammaticality asymmetry observed in the comprehension studies was due to the a priori response bias. The DDM model applied to the Marking and Morphing account of attraction predicts that as the tendency towards yes responses decreases, the effect of a plural attractor in grammatical sentences should increase. \citeand{HammerlyEtAl2019} found an increased effect of plural attractor in grammatical sentences in their Experiment 3, where they informed participants that most sentences are ungrammatical in the experiment. We argued that shifting response bias towards no responses might induce ungrammaticality illusion in addition to the grammaticality illusion in another language with a structure that was found to give rise to attraction effects \citep{LagoEtAl2019}.

To test this hypothesis, we have used experimental items from Experiment 1 and introduced a within-subject bias (bias towards grammaticality x bias towards ungrammaticality) manipulation using some instructions and the ratio of ungrammatical sentences to grammatical sentences. We only manipulated the number of ungrammatical fillers and grammatical fillers; the experimental items were the same in both within-subject conditions.

We calculated the participants' response bias using the formula provided in \citeand{MacmillanCreelman2005}. It seemed that we were not able to uniformly manipulate our participants' response bias; the the effect of instruction and the ratio of ungrammatical fillers did not create a systematic difference in participants' bias. However, when we included the calculated bias in our Bayesian GLM, we saw that the grammatical conditions with plural attractors were less likely to be judged as grammatical when participants did not have a bias toward yes responses. Our findings were parallel with \cites{HammerlyEtAl2019} findings and theoretical assumptions, even though we could not manipulate the bias properly. 

Both our and \cites{HammerlyEtAl2019} findings cannot be accounted for if we assume a cue-based retrieval account, which argues that the attractor's plurality is only relevant in ungrammatical sentences \citep{LagoEtAl2015, LagoEtAl2019, WagersEtAl:2009}. Since there would be a complete match in grammatical sentences between the cues provided by the singular verb and features in the singular subject head, a plural attractor has no way of interfering with the subject-verb dependency. Thus, the lack of an effect induced by plural attractors in grammatical sentences (grammaticality asymmetry) result from the internal mechanisms of how a cue-based retrieval system works. The response bias has no way to affect the retrieval process, and therefore, should not influence processing agreement attraction. 

The Marking and Morphing account, on the other hand, expects a comparable effect of plural attractor in both grammatical and ungrammatical sentences. According to this account, agreement attraction occurs after an erroneous representation is formed. Like grammaticality illusion, in which participants erroneously judge a grammatical sentence as an ungrammatical one, there should be an ungrammaticality illusion, which means that participants occasionally deeming grammatical sentences as ungrammatical. The non-existence of such an effect in the previous agreement attraction experiments can be explained via a response bias towards yes responses.

Even though both we and \citeand{HammerlyEtAl2019} could provide evidence for the representation-based attraction accounts, we believe that \cites{HammerlyEtAl2019} results should be verified. We used filler items to determine and check our participants' bias values. However, \citeand{HammerlyEtAl2019} used all items in their experiment. We believe that using all items might create a problematic picture since the bias calculation would also include agreement attraction effects in grammatical and ungrammatical sentences. If there is a bias towards any response type, it should also be present in fillers. When we checked participants' response bias in their Experiments 1 and 3 using their fillers, we saw that they could not manipulate their participants' bias systematically as well. Nevertheless, their Experiment 3 clearly shows an effect of a plural attractor independent of the sentence grammaticality, which might be due to their participant sample.

\section{Syntactic assumptions} \label{ch6syntax}

In Chapter \ref{ch:exp2}, we have discussed that our results from Experiment 2 might be due to syntactic depth differences. A number information coming from the verb of a relative clause, which is embedded in more phrases than the number information coming from a genitive modifier, could not induce attraction effects. The Marking \& Morphing account of agreement attraction predicts this effect of syntactic depth. In their spreading activation formula used to calculate the final number representation of a nominal phrase, the contribution of various elements in the same phrase is weighted according to their syntactic distance to the root node of the subject phrase.

Here, I repeat the structures we posited in Chapter \ref{ch:exp2}. The structure for a Genitive-Possessive DP shown in (\ref{syntax:genpossRepeat}) is adapted from \citeand{OzturkTaylan2016}. Prior to their study, many other researchers as well assumed a structure in which the genitive-marked DP starts from a position that is close to the head NP but moved up to the spec DP position to be marked with a genitive case \citep{Lewis70,Dede78, Kornfilt97, Kornfilt85,Ozsoy94, Yukseker98, Arslan2006, Arslan2009, Goksel2009}. Due to its position, the weight of the number information coming from the DP \emph{y\"oneticilerin} would be very high, and the additional number information would easily influence the final number representation.

\ea \label{syntax:genpossRepeat} {Genitive-Possessive DP}\\*
\scalebox{1}{\begin{forest}
%fned
    [DP
        [DP\\Y\"oneticilerin\textsubscript{i}]
        [DP
            [\emph{n}P
                [DP\\\emph{t}\textsubscript{i}]
                [\emph{n}P
                    [NP [N\\a\c{s}\c{c}{\i}]]
                    [\emph{n}\\-s{\i}]
                ]
            ]
            [D]
        ]
    ]
    % \path[fill=red] (.parent anchor) circle[radius=2pt];
    % \path[fill=red] (!1.child anchor) circle[radius=2pt];
\end{forest}}
\z

On the other hand, when we look at the inner syntax of a Turkish relative clause, there is yet to be a single representation that is widely assumed. The structure shown in \ref{syntax:orcRepeat} is adapted from \cites{Aygen2002} work. The relative clause is an adjunct at the DP level and consists of syntactic phrases VP, little \emph{v}P, TP, little \emph{n}P, and DP. It assumes that Turkish relative clauses are not full-CPs. This assumption follows from the fact that CP-level adverbials like \emph{Allah'tan} (\emph{Thank God}) cannot be licensed in relative clauses \citep{Goksu2018, Aygen2002}. We also assume that terminal nodes introduce full words with feature specifications and not morphemes, following \citet{C20,C21}. In this syntactic approach, morphological derivations of an utterance are completed prior to the syntactic derivations, and syntactic mechanisms check whether there is a match between the specifications given in the terminal node and the specifications in the checking node. 

\ea \label{syntax:orcRepeat} {Object Relative Clause}\\*
\scalebox{1}{\begin{forest}
%fned
    [DP
        [DP
                [\emph{n}P
                    [TP
                        [DP\\\emph{pro}\textsubscript{j}]
                        [TP
                            [\emph{v}P
                                [DP\\\emph{t}\textsubscript{j}]
                                [\emph{v}P
                                    [VP
                                        [DP\\\emph{t}\textsubscript{i}]
                                    [V\\tuttuklar{\i}\\{$[$}\Tpl{}{$]$}]
                                ]
                                [\emph{v}]
                            ]
                        ]
                        [T]
                    ]
                ]
                [\emph{n}]
            ]
            [D\\{$[$}\emph{u}\Tpl{}{$]$}]
        ]
        [DP
            [NP [N\\a\c{s}\c{c}{\i}\textsubscript{i}] ]
            [D]
        ]
    ]
    % \path[fill=red] (.parent anchor) circle[radius=2pt];
    % \path[fill=red] (!1.child anchor) circle[radius=2pt];
    % \path[fill=red] (!11.child anchor) circle[radius=2pt];
    % \path[fill=red] (!111.child anchor) circle[radius=2pt];
    % \path[fill=red] (!1112.child anchor) circle[radius=2pt];
    % \path[fill=red] (!11121.child anchor) circle[radius=2pt];
    % \path[fill=red] (!111212.child anchor) circle[radius=2pt];
    % \path[fill=red] (!1112121.child anchor) circle[radius=2pt];
    % \path[fill=red] (!11121212.child anchor) circle[radius=2pt];
\end{forest}}
\z

In our case, the terminal V node introduces the word \emph{tuttuklar{\i}} which comes with an agreement feature [\Tpl{}] in addition to case, tense, and aspect features. This feature will later check the uninterpretable feature [\emph{u}\Tpl{}] under the D head. The model assumes that the syntactic tree will be sent to the semantic-computation interface, and this interface cannot work with uninterpretable features. Thus, all uninterpretable features must be checked. The most important aspect of this analysis is that the features are introduced in the terminal node and the hierarchically upper nodes only do the checking job. 

Consider another possible analysis which does not share the same assumptions with the previous model of syntactic theory. In this set of analyses, the full form of the words is not provided in one single node. Instead, different morphemes are provided in various syntactic nodes depending on their semantic content. Theories like Distributed Morphology \citep{HarleyNoyer99,HalleMarantz94} and Nanosyntax \citep{Starke2010, Taraldsen2010, Caha2009} used this type of analysis extensively. (\ref{syntax:orcDM}) shows another way to represent object relative clauses in Turkish. We also repeat the genitive-modified noun phrases to show the comparison of syntactic depth. We also provide the inner syntactic structure of the attractor DP.



    \ea 
    \ea \label{syntax:orcDM} {Object Relative Clause}\\*
    \scalebox{0.9}{\begin{forest}
        %fned
            [DP
                [DP
                        [\emph{n}P
                            [TP
                                [DP\\\emph{pro}\textsubscript{j}]
                                [TP
                                    [\emph{v}P
                                        [DP\\\emph{t}\textsubscript{j}]
                                        [\emph{v}P
                                            [VP
                                                [DP\\\emph{t}\textsubscript{i}]
                                            [V\\tut-]
                                        ]
                                        [\emph{v}]
                                    ]
                                ]
                                [T]
                            ]
                        ]
                        [\emph{n}\\-tuk]
                    ]
                    [D\\-lar{\i}]
                ]
                [DP
                    [NP [N\\a\c{s}\c{c}{\i}\textsubscript{i}] ]
                    [D]
                ]
            ]
            \path[fill=black] (.parent anchor) circle[radius=2pt];
            \path[fill=black] (!1.child anchor) circle[radius=2pt];
            \path[fill=black] (!12.child anchor) circle[radius=2pt];
        \end{forest}}
    
    \ex \label{syntax:genpossRepeat2} {Genitive-Possessive DP}\\*
    \scalebox{0.9}{\begin{forest}
        %fned
            [DP
                [DP\textsubscript{i}
                    [PlP
                        [NP [NP\\Y\"onetici] ]
                        [Pl\\-ler]
                    ]
                    [D\\-in]
                ]
                [DP
                    [\emph{n}P
                        [DP\\\emph{t}\textsubscript{i}]
                        [\emph{n}P
                            [NP [N\\a\c{s}\c{c}{\i}]]
                            [\emph{n}\\-s{\i}]
                        ]
                    ]
                    [D]
                ]
            ]
            \path[fill=black] (.parent anchor) circle[radius=2pt];
            \path[fill=black] (!1.child anchor) circle[radius=2pt];
            \path[fill=black] (!11.child anchor) circle[radius=2pt];
            \path[fill=black] (!112.child anchor) circle[radius=2pt];
        \end{forest}}
    \z
    \z
    

Unlike the previous syntactic theories in which the plural information is introduced under the V-head within the relative clause, the plural information is introduced in a relatively higher position in (\ref{syntax:orcDM}). In this type of representation, we do not utilize the checking theory, and every node, or set of nodes, spells out the morphological counterpart of the function they serve.  For example, the past tense inflection (\emph{-ed}) in verbs like \emph{jumped} would reside in the T head in this set of theories, whereas it would reside under the V head with approaches that utilize the Checking Theory.

As you can see in syntactic trees (\ref{syntax:orcDM}) and (\ref{syntax:genpossRepeat2}), the plural information in the Genitive-Possessive DP construction is embedded more deeply than the plural agreement marking in the relative clause construction. According to the spreading activation formula of Marking \& Morphing theory, the contribution of the plural marking in (\ref{syntax:orcDM}) to the root node should be higher since its weight which is determined according to their syntactic depth will be higher. 

Even though one may try to compare these two types of theories and conclude that the former explains our results better, we are deliberately avoiding this conclusion. This brief discussion did not aim to argue for what a better syntactic theory should be. Instead, it aimed to show that there must be certain assumptions about syntactic representation that we need to be explicitly utter. According to the syntactic assumptions, the predictions of the Marking \& Morphing account might have conflicting results. We assumed a model that introduced the whole words under the V nodes in this thesis. 

\section{Honorific reading and agreement attraction} \label{ch6hon}

Another alternative explanation for the initial agreement attraction findings that we have not covered in this thesis is a possible honorific/formal reading, which might satisfy the presence of a plural marking at the verb. As discussed in Chapter \ref{ch:intro}, not all plural markers on the verb are number agreement markers in Turkish \citep{GokselKerslake2005}. Consider sentences in (\ref{ex:larPoliteConc}). The sentence is ungrammatical with the intended meaning of plural number agreement. However, the sentence is grammatical if we assume a formal register. In a context where we utter this sentence to a person who is socially or hierarchically higher than us, the sentence is perfectly fine. We can continue this sentence with phrases like \emph{sir} (\emph{efendim}) as in (\ref{ex:larSir}),  but not with phrases like \emph{lan} as in (\ref{ex:larLan}).

\ea 
    \ea[]{\label{ex:larPoliteConc}
    \gll Doktor Han{\i}m gel-di-ler.\\
    doctor Ms. come-\Pst{}-\Hon{}/*\Tpl{}\\
    \glt `Ms. Doctor has arrived.' \\* * `Ms. Doctor have arrived.'}
    \ex[]{\label{ex:larSir}
    \gll Doktor Han{\i}m gel-di-ler efendi-m.\\
    doctor Ms. come-\Pst{}-\Hon{}/*\Tpl{} sir-\Fpl.\Poss{}\\
    \glt `Ms. Doctor has arrived, sir.'}
    \ex[*]{\label{ex:larLan}
    \gll Doktor Han{\i}m gel-di-ler lan.\\
    doctor Ms. come-\Pst{}-\Hon{}/*\Tpl{} yo\\
    \glt `Yo, Ms. Doctor has arrived.'}
    \z
\z

We hypothesized that due to the nature of complex noun phrases we and \citeand{LagoEtAl2019} utilized, the honorific reading might be the underlying reason for the presence of agreement attraction. The relationship between the attractor and the head noun was always a job-related relation: either the attractor provided a professional service to the head noun as in \emph{managers' cook} or the head was superior to the attractor \emph{students' professor}. Therefore, on some occasions, participants might entertain a formal context which can prevent the sentence from crashing even if it is ungrammatical in informal contexts.

To test this possibility, we conducted a speeded acceptability judgment task in which we manipulated the number of the attractor (singular x plural), the number of the verb (singular x plural), and the post-verbal register marker (sir x yo). The head subject was always singular. One example of experimental conditions can be seen in (\ref{ex:exp4}). The conditions are provided with slashes and curly braces.

\ea \label{ex:exp4}
    \gll [{Milyoner-\O/ler-(n)in} {terzi-si}] tamamen gereksizce kov-ul-du-\O/lar lan/efendi-m.\\
    millionaire-\{\Sg/\Pl\}-\Gen{} tailor-\Poss{} completely without.reason fire-\Pass-\Pst-\{\Tsg{}/\Tpl{}\} \{yo/sir-\Fpl.\Poss{}\}\\
    \glt `\{Sir/Yo\}, the \{millionaire's/millionaires'\} tailor \{was/were\} fired for no reason at all.'
\z

Our results showed that the presence of a formal register overall increased the acceptability of ungrammatical sentences. However, a plural attractor was present in formal and informal registers when the sentence was ungrammatical. If initial attraction findings in Turkish were due to a possible honorific reading of the \emph{-lar} marking on the verb, we would expect to have an increased overall acceptability with plural attractor in ungrammatical sentences only in the formal register conditions. However, this is not the case. For a detailed explanation and analysis of this experiment, see Appendix \ref{ap:exp4}.

\section{General discussions}

Overall, our findings suggest that participants did not utilize form-related cues that are either introduced with the ambiguous case markers on the subject head or the homophonous markers of plurality and \Tpl{} agreement. The previous findings of \citeand{LagoEtAl2019} were a genuine case of agreement attraction. Agreement attraction effects were not due to case syncretism, lingering effects of erroneous parse, or task-specific response strategies. 

Existing cue-based retrieval accounts cannot explain our findings since most previous studies and theorization do not refer to the role of part-of-speech tags and case syncretism. Cue-based retrieval would expect a reduced effect of plural attractor in ungrammatical sentences when the case syncretism is eliminated (Experiment 1). In addition, all previous research on agreement attraction that dealt with case syncretism, and found significant effects, manipulated the syncretism on the attractor. Our results show that being head matters in the interaction between case marking and attraction effects. The promotion of the head role in sentence processing cannot be accounted for via cue-based retrieval theories without any additional assumptions.

Similarly, the results of Experiment 2 cannot be explained via cue-based retrieval accounts. These models would expect interference due to the shared form of plural and agreement marking. However, our results showed that even nominalized verbs could not induce agreement attraction effects. Cue-based theories would need to assume that there should be two different number features: one for agreement and one for plurality. It would also need to keep record of part-of-speech tags and entertain only the chunks that are marked with a denominal feature.

We were also able to replicate theoretical implications of \citeand{HammerlyEtAl2019}, which argued that grammaticality asymmetry is due to the a priori response bias, not the retrieval mechanisms. We showed that participants' bias affected the attraction patterns. Participants accepted not only ungrammatical sentences with plural attractors more often than the singular attractor ones but also grammatical sentences with plural attractors compared to their singular attractor counterparts. Results of Experiment 3 posed another challenge for cue-based retrieval theories: an interference of an irrelevant cue (+\Pl{}) when there is a full match between the cues and the features (+\Sg{}, +{subj}). 

Taken together, our results can be explained via the Marking and Morphing account of agreement attraction. Due to the lack of specification of any mechanism that incorporates case-marking in the Marking and Morphing account, we would expect no difference in attraction patterns when the local ambiguity due to the case syncretism was not present. Moreover, since the contribution of a plural diminishes depending on its syntactic depth, the Marking and Morphing account would predict a reduced or no effect of plural attractor in relative clause constructions. Lastly, an effect of the presence of plural attractors independent of sentence grammaticality is one of the signature predictions of the Marking and Morphing account. We showed that its predictions hold when the extra-linguistic factors, such as response bias, is nullified. 

In the future, it would be interesting to calculate participants' bias in previous agreement attraction experiments and present a meta-analysis to investigate whether the previous patterns of acceptability in grammatical sentences were due to the response bias. Moreover, a notional replication of our Experiment 2, where the head subject is marked with overt possessive rather than a nominative, would provide a healthier comparison between relative clause attractors and genitive-modifier attractors.\footnote{We used object relative clauses in our Experiments 2A and 2B. The head subject was a bare DP in all experimental sentences with a relative clause, different from our other experiments in which the head subject was marked with an overt possessive marking. One can circumvent this problem by using complement clauses as in (\ref{ex:complementCP}). 

\ea \label{ex:complementCP} 
\gll Gel-dik-ler-i haber-i h{\i}zl{\i} duy-ul-du.\\
come-\Nmlz{}-\Tpl{}-\Poss{} news-\Poss{} fast hear-\Pass{}-\Pst{}\\
\glt `News of their coming was heard fast.'
\z

However, since complement CPs can only be used with inanimate nouns like news, gossip, or story, we would need to have another baseline attraction experiments with inanimate subjects.} Lastly, we believe that we need to replicate Experiment 1 with a better set of fillers since the number of ungrammatical items might affect the participants' response bias, thus the attraction patterns. 
