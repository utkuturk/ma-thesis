\chapter{INTRODUCTION} \label{ch:intro}

In this chapter, I will present the aim of this thesis, the linguistic conventions used throughout the thesis, the statistical approach, and some basic properties of Turkish that will be necessary for the remainder of the thesis. I will also give the outline of the thesis.

\section{Aim of the thesis}

This thesis explores the processing of subject-verb number agreement. Specifically, it investigates the sentences in which there is an additional plural-marked element, attractor, and how it interferes with the subject-verb number dependency. The typical example for this interference called number agreement attraction can be seen in (\ref{ex:typicalExampleSgPl}) and (\ref{ex:typicalExamplePlPl}) taken from \citeand{BockMiller:1991}.

\ea \label{ex:typicalExample}
  \ea * The key to the cabinet\phantom{{s}} are on the table. \label{ex:typicalExampleSgPl}
  \ex * The key to the cabinet{s} are on the table. \label{ex:typicalExamplePlPl} 
  \z
\z

Previous research has found that participants find ungrammatical sentences acceptable more often and have less difficulty processing them when there is an additional plural marked element, attractor, in the vicinity as in (\ref{ex:typicalExamplePlPl}) compared to (\ref{ex:typicalExampleSgPl}). Examples in (\ref{ex:typicalExample}) are essential for the following reasons. Both sentences are ungrammatical because the agreement controller {\it key} is singular, but the verb {\it were} is plural. However, the degree of perceived ungrammaticality, thus their acceptability, differs from one another. While the ungrammaticality in (\ref{ex:typicalExampleSgPl}) is easily noticed, psycholinguistic studies have shown that people systematically fail to see the ungrammaticality in (\ref{ex:typicalExamplePlPl}) (Nicol, Forster, \& Veres, 1997; Pearlmutter, Garnsey, \& Bock, 1999; Wagers, Lau, \& Phillips, 2009). This difference in acceptability was found to be robust both in production \citep{BockMiller:1991} and comprehension \citep{NicolEtAl1997,PearlmutterGarnseyBock:1999} of such sentences in various languages, including Arabic (Tucker, Idrissi, \& Almeida, 2015), Armenian (Avetisyan, Lago, \& Vasishth, 2020), Hindi \citep{BhatiaDillon:2020}, Spanish (Lago, Shalom, Sigman, Lau, \& Phillips, 2015), and Turkish \citep{LagoEtAl2019}. 

Within the last 30 years, researchers have found that an effect and its magnitude are contingent on various syntactic, semantic, and extra-linguistic factors. These factors include syntactic distance effects (Hartsuiker, Ant{\'o}n-M{\'e}ndez, \& Van Zee, 2001; Nicol et al., 1997; Kaan, 2002), linear distance effects \citep{Pearlmutter2000,BockCutting1992}, the effects of syncretic forms \citep{Slioussar2018}, distributivity characteristics and collective readings of nouns involved (Eberhard, 1999; Vigliocco, Hartsuiker, Jerema, \& Kolk, 1996; Kurtzman \& MacDonald, 1993; Humphreys \& Bock, 2005), syntactic category of the phrase containing the attractor \citep{BockMiller:1991,BockCutting1992}, a priori response bias of the participants (Hammerly, Staub, \& Dillon, 2019), and more.

These findings have been accounted for mainly via three different accounts: (i) feature percolation, (ii) marking and morphing, and (iii) cue-based retrieval. 


Feature percolation accounts started with the pioneering work done by \citeand{BockMiller:1991}. Bock and her colleagues proposed a theory of agreement attraction that speculates that some features of the attractors are percolated upwards to the agreement controller \citep{BockMiller:1991,BockCutting1992,BockEberhard93}. In structures such as `\emph{the key to the cabinets \ldots{}},' the plural feature of the attractor \emph{cabinets} migrated or copied to the higher element, the agreement controller \emph{key}.  This understanding of agreement attraction is closely related to the notions of feature inheritance and feature copying from the prominent syntactic theory of generative syntax (Chomsky, 1993; Gazdar, Klein, Pullum, \& Sag, 1985). Similar to these notions, the number feature of the plural attractor may be copied to the syntactically dominating singular controller, which in turn erroneously licenses an agreement between the singular agreement controller and the plural verb, agreement probe. 

However, many studies have found that a syntactic relation, such as sharing the root node in phrase between,the controller and the attractor is not needed for attraction effects to surface (Hartsuiker et al., 2001; Franck, Lassi, Frauenfelder, \& Rizzi, 2006; Pfau, 2003). An example of such an agreement attraction phenomenon can be seen in (\ref{ex:MMreasons_object}). The direct object in the sentence \emph{de monteurs} interfered with the agreement process between the auxiliary \emph{hebben} and the subject \emph{de baas}. In addition, attraction rates were found to be affected by the semantic manipulations such as the distributive reading of the distractor as in (\ref{ex:MMreasons_semantic}) as opposed to (\ref{ex:MMreasons_semantic2}) even though both have the same syntactic structure (Vigliocco, Butterworth, \& Semenza, 1995; Eberhard, 1997; Humphreys \& Bock, 2005). 

\ea \label{ex:MMreasons}
  \ea[*]{\label{ex:MMreasons_object} 
  \gll Peter roept dat de baas de monteurs hebben gebeld.\\
  Peter shouts that the boss the mechanics have called\\
  \glt `Peter shouts that the boss have called the mechanics.'}
  \ex[]{The gang on the motorcycles \ldots \label{ex:MMreasons_semantic}}
  \ex[]{The gang near the motorcycles \ldots \label{ex:MMreasons_semantic2}}
  \z
\z

The fact that syntactically unrelated distractors and semantic notions such as distributivity and collective readings could probe attraction effects pointed towards a more forgiving analysis in terms of the limitations on the percolation. The Marking and Morphing account argued that features could percolate between any syntactic nodes; however, the syntactic distance these features need to move reduces the possibility of attraction as it increases (Eberhard, Cutting, \& Bock, 2005). In addition, the number attraction may also occur in the notional representation level, which is independent of the syntax. The agreement has two different stages in this model: Marking and Morphing. At the number-marking stage, participants form a conceptual representation of the phrase. A notional plurality of an expression of the available distributive readings may result in agreement attraction effects in the number-marking stage. In addition to the number-marking stage, attraction can also occur in the number-morphing stage. In this stage, the attraction is governed by other sources of number information and their syntactic distance to the subject head. A new number value is given to the whole phrase with the notional number and the weighted numbers of other elements in the sentence. If this new number is not definitively singular, then the attraction may surface. The magnitude of the effect is conditional on the aforementioned pieces of information.

The Marking and Morphing account handles issues such as distributivity, interference of direct objects, and attractors such as {\it gang}, which are syntactically singular but notionally plural. However, the fact that these effects were usually seen in ungrammatical sentences as in (\ref{ex:asymmetryExamplePl}) but not in (\ref{ex:asymmetryExampleSg}) could be explained by neither feature percolation nor the Marking and Morphing accounts \citep{WagersEtAl:2009}. 

\ea \label{ex:asymmetryExample}
  \ea[]{The key to the cabinets was rusty. \label{ex:asymmetryExampleSg}}
  \ex[*]{The key to the cabinets were rusty. \label{ex:asymmetryExamplePl}}
  \z
\z

An account of attraction based on the cue-based retrieval \citep{LewisVasishth:2005} successfully explained these facts. These accounts theorize that the attraction occurs after the verb is read, and it is due to an erroneous retrieval of the agreement controller, and not due the erroneous representation. When the sentence is grammatical, as in (\ref{ex:asymmetryExampleSg}), the cues of the verb completely match with the features of the subject. Due to this total match, the features of the attractor cannot interfere with the subject-verb dependency and affect the processing. However, in ungrammatical sentences like (\ref{ex:asymmetryExamplePl}), there is no single total match, and both nouns match partially with the cues. The attractor \emph{cabinets} matches the number feature, and the head \emph{key} matches the subjecthood related feature. Thus, both nouns compete to resolve the dependency relation. According to retrieval accounts, participants' memory falters occasionally, and the verb erroneously agrees with the attractor on those occasions. Thus, the attraction results from a memory-fallacy, not a representation-related problem. However, a recent study by \citeand{HammerlyEtAl2019} showed that this grammaticality asymmetry could be explained via response bias and not necessarily due to memory-retrieval processes. 

There are additional accounts that incorporate focuses on (i) rational interference (Ryskin, Bergen, \& Gibson, 2021), (ii) competition \citep{NozariOmaki2022}, and (iii) self-organized sentence processing (SOSP) (Villata, Tabor, \& Franck, 2018; Smith, Franck, \& Tabor, 2018, 2021). According to the rational interference account, participants consider the probability of an utterance given a language model and the likelihood that noise corrupted the originally intended sentence into the utterance they encountered. When participants find corruption more likely to happen than the sheer ungrammaticality, they correct the utterance they encounter; thus, agreement attraction effects arise. As for the competition model, \citeand{NozariOmaki2022} assumes that every pre-verbal plural element activates the plural verb form, and activation is directly contingent on how recently it was produced. Lastly, in SOSP models, the minimal unit of operation is a treelet. These treelets combine with other treelets depending on how well their features match each other. When there is more than one possible way to form treelets, competition arises among them, creating processing difficulty and slowing the processing, thus the attraction effects. 

To sum up, there is no consensus of what is the underlying nature of the attraction effects. Most of the theorization depends on a limited number of experiments in limited number of languages, which creates an opportunity to investigate different languages using different constructions with different manipulations. By exploring the murkier areas in the attraction field, we hope to provide additional emprical data and clear picture of the attraction. 

The main aim of this thesis is three-fold: (i) to investigate the role of local ambiguities, shallow processing, and response bias, as well as to eliminate possible confounds in the previous findings, (ii) to contextualize the findings on task effects within the existing agreement attraction accounts, and (iii) to present a comprehensive picture of Turkish agreement attraction facts. To this end, we conducted four speeded acceptability judgment experiments using sentences based on \cites{LagoEtAl2019} items. An exemplary structure is shown in (\ref{ex:lagoTease}). Attractors in our Turkish items always precede the head, and the number is marked in an agglutinative manner overtly with the suffix \textit{-lAr}.\footnote{A in \emph{-lAr} is an archiphoneme. Archiphonemes are used when the sound is underspecified for certain features. Throughout the thesis, we make use of archiphonemes. A stands for non-high vowels which are underspecified in their backness feature. I stands for high vowels which are underspecified in both backness and roundness features. Thus, \emph{-lAr} means that the suffix may either surface as \emph{-ler} or \emph{-lar} depending on the previous vowel. Similarly, the possessive suffix \emph{-sI(n)} may surface as one of the following forms: \emph{-s{\i}}, \emph{-si}, \emph{-su}, \emph{-s\"{u}}.}

\ea[*]{\label{ex:lagoTease}
\gll {Milyoner-ler-in} {terzi-si} tamamen gereksizce kov-ul-du-lar.\\
millionaire-\Pl-\Gen{} tailer-\Poss{} completely without\_reason  fire-\Pass-\Pst-\Tpl.\\
\glt `The millionaires' tailor were fired for no reason at all.'}
\z

Experiment 1 (see Chapter \ref{ch:exp1} for details) investigates a possible confound in \cites{LagoEtAl2019} items and the effects of local ambiguity caused by a case syncretism. Since all subject heads in the \cites{LagoEtAl2019} study end with a consonant, the marking on the subject head is ambiguous between the possessive and accusative suffix. We modified \cites{LagoEtAl2019} items, used unambiguous subject heads with unambiguous possessive marking, and replicated the \cites{LagoEtAl2019} experiment.

Experiments 2A and 2B (see Chapter \ref{ch:exp2} for details) explore a possible explanation for agreement attraction based on shallow-processing. Turkish verbal and nominal plural morphemes are identical, unlike other languages where the agreement attraction effects are seen. Due to this fact, we hypothesized that previous findings might be due to a shallow-parsing mechanism where participants check whether or not there was a plural marking present in the sentence and deem sentences grammatical if they have a memory of the form of the plural morpheme. 

Experiment 3 (see Chapter \ref{ch:exp3} for details) is concerned with a priori response bias of participants and ungrammaticality illusion. A recent study by \citeand{HammerlyEtAl2019} showed that an important generalization, grammaticality asymmetry, can be modeled as a function of the response bias in the psycholinguistic experiments, rather than a side effect of a reanalysis process as proposed by the cue-based retrieval theory. Since their findings challenge the semi-established understanding of agreement attraction and are only shown in one language using one structure, we wanted to replicate their results in Turkish. Given that the basic assumptions of response bias analysis and the Marking and Morphing account should not depend on a specific language, we expect to see similar effects of the plural attractor both in grammatical and ungrammatical sentences.

\section{Linguistic conventions}

Throughout the thesis, we gloss linguistic examples using Leipzig glossing conventions (Haspelmath, 2014; Comrie, Haspelmath, \& Bickel, 2008) and use capital letters to indicate allomorphy. We use the Modern Standard Turkish orthographic conventions for linguistic examples in which most, but not all, letters match with IPA symbols. The following is the IPA counterparts of non-comforting sounds: \"u for [y], \"o for [\textipa{\o}], {\i} for [\textipa{W}], \c{c} for [\textteshlig], c for [\textdyoghlig], \c{s} for [\textesh]. All decomposable morphemes  are separated by a dash `$-$' both in the example and in the glossing line. Non-decomposable and zero morphemes are only shown in the glossing line with a dot `$.$' and square brackets, respectively. All ungrammatical sentences are marked with an asterisk `$*$' at the beginning of the sentence, while grammatical ones are not marked with any symbol. If there is a speaker variability, we used the percentage symbol `$\%$'. When we want to emphasize a feature or when the language does not have a morphological output for a specific feature, we use a subscript text to highlight this feature or show the abstract feature. For example, the number information in English sentences with past tense is not shown explicitly, so we sometimes mark it with a subscript text.

\subsection{Terminology: acceptability versus grammaticality}

The central claims of this thesis stand on top of notions like grammaticality and acceptability, which are used in the thesis reasonably often. We adopt an understanding in which we consider acceptability as a perceived grammaticality in the lines of \citeand{C65}. Thus, acceptability reflects both the internal grammar of a speaker and performance factors such as memory, bias, noise, difficulty in parsing, or competing parses.

The grammaticality of a statement depends on whether a statement can be generated by a specific grammar or not. It can be represented in a binary (grammatical versus ungrammatical), in a gradient manner using combinations of $?$ and $\ast$ symbols ($?<{\ }??<{\ }?\ast<\ast<\ast\ast$) similar to ordinal scales (5/7-Point Likert Scale), or in a fully gradient manner \citep{Keller2000}. It is important to note that the given grammaticality status of a statement is based on a speaker's personal judgments and intuitions, which follow from the characteristics of I-language: individual and internal \citep{C86}.

On the other hand, the acceptability of a statement is dependent on many factors, only one of which is grammaticality \citep{C65}. Sentences generated by grammar may be deemed unacceptable by the active users of the same grammar. Depth-3 center-embedded sentences are among such sentences \citep{GibsonThomas99}. Even though sentences like `\emph{[The patient [who the nurse [who the clinic had hired] admitted] met Jack]},' are grammatical, their processing is complicated, which results in reduced acceptability. On the other hand, ungrammatical sentences may be found acceptable under certain conditions. Examples for such instances, such as `*\emph{[The patient [who the nurse [who the clinic had hired] ] met Jack]},' are provided by \citep{GibsonThomas99}. The sentence provided here is ungrammatical since the second verb is missing. However, native speakers found these sentences acceptable in a reasonably systematic fashion. 

This thesis is mainly concerned with acceptability and how certain linguistic and non-linguistic phenomena affect the perceived grammaticality, that is, acceptability. We are more interested in performance mechanisms rather than competence mechanisms. We measure the acceptability of the sentences in our experiments, not the grammaticality. The inferences we make are about the interaction between a speaker's grammar and other features like bias, memory, or noise.

However, we need to quote \citeand{PhillipsLasnik03} here: experimental findings in linguistics affected generative linguistics and the understanding of the internal grammar. In topics such as merge \citep{Levelt74}, long-distance dependencies through the lens of movement and traces (Fodor, Bever, \& Garrett, 1974), and argument structures \citep{Pinker1989}, psycholinguistic studies paved the way for later theorization of the internal machinery of language.

Despite this relation between the previous psycholinguistic studies and acceptability findings, we prefer using the word acceptability and not grammaticality in our methodology. Nevertheless, there are three primary contexts we deliberately used the word grammaticality: (i) grammaticality asymmetry, (ii) grammaticality illusion, and (iii) grammaticality as a predictor. In all of these instances, the main rationale behind using grammaticality notion rather than acceptability was our belief that the competent-related state of an expression had a defining effect on the overall acceptability. 

For example, in the case of the grammaticality illusion, the presence of a plural attractor gave the illusion of grammaticality in ungrammatical sentences while not giving the illusion of ungrammaticality in grammatical sentences. This illusion cannot be reworded using acceptability since the way we estimate illusion is the increased acceptability of ungrammatical sentences and the reduced acceptability of grammatical sentences. We assume that in a perfect noise-free environment with bias-free and limitless-memory participants, we would have acceptability rates close to 0 in ungrammatical sentences.  

\section{Experimental details}

All experiments in this thesis are speeded acceptability judgments with forced binary good and bad options. Even though speeded acceptability judgments have some limitations, they are found to be reliable, replicable, and easily applicable to decision-making and memory theories. 

\citeand{BaderHaussler2010} showed that speeded acceptability judgments provided comparable results across different modalities of acceptability ratings, such as unspeeded Magnitude Estimation. Despite having completely different time pressures and measurement methods, binary choices in a speeded environment were quantitatively similar to methods that provided a more detailed picture.

It is important to note that the effect size is a significant factor here. \citeand{SprouseAlmeida2017} showed that with a good pool of participants, binary yes-no tasks are powerful enough to provide insights for effects bigger than $d<.5$ following \cites{Cohen92} criteria. They provide Bayes Factor tests displaying that Binary Yes-No experiments had comparable power with experiments that employ Magnitude Estimation, Likert Scale, or Forced-Choice methods. Even though Binary Yes-No experiments require more participants than the rest of the methods and they only achieve enough power with medium ($.5<d<.8$), large ($.8<d<1.1$), and extra large ($1.1<d$) effect size, we found this methodology appropriate for our study since agreement attraction effect sizes are not generally small.


\section{Statistical choices}

\subsection{Why Bayes?}

In this thesis, we make use of Bayesian inference. There are multiple reasons behind this choice. 

Bayesian inference allows us to integrate our beliefs and hypotheses into the data analysis process. It is done using prior distributions $P(\theta)$ in formula (\ref{bayesformula}). While the likelihood part $(P(y_{i}|\theta))$ depends solely on the data itself and expresses how likely is the data point ($y_{i}$) is given our hypothesis ($\theta$), the prior part $P(\theta)$ gives us the prior possibility of our hypothesis ($\theta$). In Bayesian inference, we multiply every data point with a probability distribution that we specify according to what we believe is going on in the world. By doing so, we give our hypotheses definite forms and allow us to formulate possible competing explanations of the data and test both of them against the data. 

\begin{equation}\label{bayesformula}
  P(\theta|y) \propto \prod_{i}^{N} P(y_{i}|\theta) P(\theta)
\end{equation}


This procedure also allows us to decide how much we want to integrate from previous literature, which is made possible by the use of priors. In addition to our hypotheses, we can inform our model and calculations about previous behavioral data. For example, response times typically have a positive skew with a long tail following the central mass, as stated in \citeand{LeeEtal2018} and \citeand{Luce1986}. Specifying this tendency in a model would deem some response time values less likely, and thus would diminish the effect of an outlier data point in our model. This also entails that not all experimental data are equal, and their contributions are equal. 

Moreover, the details of the prior distribution reflect our degree of confidence in that hypothesis. We can provide a very specific distribution with thin tails, which would mean that we are very confident about how the data is distributed. On the other hand, we can have a completely flat distribution, meaning that we have no information or prior evidence about the data. 

Lastly, it deals with uncertainty, which is an important aspect when we cannot gather all the possible data. If we were to use frequentist analyses and provide p-values in our models, we would have no way of knowing whether or not our p-value is a result of our sample size or the effect size. That is, having a small effect in magnitude and a large pool of participants and a larger effect in magnitude with fewer data points may give us the same p-value as a result. Thus, reported p-value would either tell us we have pinpointed a nice effect or we do not have enough participants. On this negative aspect of reporting p-values, a recent study has shown that when the power of the study is low, and the study has found an effect, the effect is overestimated and depicts an exaggerated picture of the phenomenon (Vasishth, Mertzen, J\"ager, \& Gelman, 2018). Using Bayesian Inference, we are not dealing with the significance filter that depends solely on the p-value. Instead, we report the posterior probability distributions for each parameter in our model, which shows the relative likelihood of any data point given our model, data, and the prior. 

\subsection{Preprocessing}
Before the Bayesian analysis, we cleaned the data and visualized general tendencies present in the data as summary plots using the tidyverse package system in R \citep{tidyverse}. 

In the data-cleaning process, we had several criteria for exclusion. The first criteria was participants' native language: we excluded participants whose native language is not Turkish. The second criteria was their accuracy in practice items: if they give wrong answers to more than half of the questions, we excluded them from the analysis. We also excluded participants that answered the questions too fast, that is below 200 milliseconds. Finally, we excluded participants with too many inaccurate answers in control conditions. 

We did not include missing data points or exclusions in our analysis and assumed that data were missing completely at random \citep{VanBuuren2018}. In this thesis, we do not report the rates of missing data, but our raw data is available.

\subsection{Bayesian modeling}

While using Bayesian Inference, we fitted models using the brms package in R \citep{R-brms_a, R-brms_b}. It allowed fitting complex hierarchical Bayesian models with five lines of code. Prior to modeling, we had to define relations between the levels of our manipulations. For example, in all experiments, we manipulated the number of the attractor; it is either plural or singular. We redefined being plural as +0.5 and being singular -0.5, which is called sum contrasts. \citeand{BrehmAlday2020} shows why setting your contrasts and specifying them explicitly is essential.

\subsection{Prior selection}

As for priors, we used weakly informative priors in our models similar to the ones provided in Stan Wiki on Github (Betancourt et al., 2020). In their blogpost, Betancourt et al. (2020) explains five different levels of priors: 
(i) flat priors, 
(ii) super-vague priors, 
(iii) weakly informative priors, 
(iv) generic weakly informative priors, 
(v) specific informative priors. 
In this context, a prior is considered informative or weak depending on its effect on the likelihood. Suppose the likelihood dominates the results, and the effect of a prior is either zero or unnoticeable. In that case, the prior is not informative. In our case, we chose priors that diminish the probability space fairly. We used a \emph{Normal}(0,1) prior for the intercept and a \emph{Normal}(0,1) prior for most of the slopes except for ungrammaticality and the interaction between ungrammaticality and the plural attractor. We set a \emph{Normal}(-4,1) prior for the ungrammaticality and a \emph{Normal}(1,0.5) prior for the interaction between the ungrammaticality and the plural attractor. These priors were set following previous findings. Lastly, \emph{Cauchy\textsuperscript{+}}(0,1) prior that is truncated at 0 for the standard deviations of random effects, and a \emph{LKJ}(2) prior for correlation matrix for the random effects are used.


\subsection{Plotting}

In summary plots, we visualized mean values and \%95 confidence interval values for our data using the ggplot2 package \citep{ggplot}. When reading summary plots, we are mainly interested in whether or not confidence intervals overlap or not. Our confidence intervals were computed following \citeand{Morey2008} and his correction of \citeand{Cousineau2005}. The reason for using these computed CIs instead of just standard errors is to include uncertainty due to sampling between different groups observed. We also assessed the variance in the difference between the two conditions. \citeand{Cousineau2005} recommends using each group's standard deviation in calculating the CIs. Moreover, we multiplied our intermediate number with $1.98$ to achieve \%95 CIs. 

In posterior plots, we visualized the mean of Bayesian model coefficients. We included \%50 and \%90 posterior intervals, and the probability of each coefficient to be smaller than $-0.1$ or bigger than $0.1$, which are Region of Practical Equivalence Region borders \citeand{K18}. This ROPE region indicates no practical effect of a coefficient. If a distribution is completely outside this area, we can say we have definitive evidence for an effect. If it covers the practical equivalence area, we can say that according to our data, there seems to be no evidence for an effect. On occasions in which only a part of the distribution resides in the area, we explicitly quantify our degree of belief towards an effect. 

In this thesis, we always fit the yes responses to our stimuli. Negative values indicate a decreasing effect on the average number of yes responses. In contrast, positive values indicate an increase in the average number of yes responses.

\subsection{Packages}

The following list is all of the software and packages we used in this thesis: 
R \citep[v4.0.3;][]{R-base} and the R-packages 
bayesplot (V1.8.0; Gabry, Simpson, Vehtari, Betancourt, \& Gelman, 2019),
brms \citep[v2.14.4;][]{R-brms_a, R-brms_b}, 
cowplot \citep[v1.1.1;][]{R-cowplot}, 
data.table \citep[v1.14.2;][]{R-data.table}, 
dplyr \citep[v1.0.8;][]{R-dplyr}, 
gdata \citep[v2.18.0;][]{R-gdata}, 
gganimate \citep[v1.0.7;][]{R-gganimate}, 
ggdist \citep[v2.4.0;][]{R-ggdist}, 
ggplot2 \citep[v3.3.5;][]{R-ggplot2}, 
ggstatsplot \citep[v0.8.0;][]{R-ggstatsplot}, 
here \citep[v1.0.1;][]{R-here}, 
knitcitations \citep[v1.0.12;][]{R-knitcitations}, 
knitr \citep[v1.37;][]{R-knitr}, 
magrittr \citep[v2.0.2.9000;][]{R-magrittr}, 
papaja \citep[v0.1.0.9997;][]{R-papaja}, 
patchwork \citep[v1.1.1;][]{R-patchwork}, 
purrr \citep[v0.3.4.9000;][]{R-purrr}, 
Rcpp \citep[v1.0.8;][]{R-Rcpp_a, R-Rcpp_b}, 
rstan \citep[v2.21.2;][]{R-rstan}, 
StanHeaders \citep[v2.21.0.7;][]{R-StanHeaders}, 
tidybayes \citep[v2.3.1;][]{R-tidybayes}, 
tidyr \citep[v1.1.3.9000;][]{R-tidyr}, 
tinylabels \citep[v0.2.1;][]{R-tinylabels}, and 
yaml \citep[v2.2.2;][]{R-yaml}.


\section{Turkish facts}

This thesis deals with the agreement attraction facts in Turkish, an agglutinative language with rich morphology. Our manipulations make us of various aspects of Turkish morpho-syntax. These include case marking, possession marking, number marking, and the relative clause structure. In this section, we briefly exemplify these aspects of Turkish morpho-syntax.

\subsection{Number agreement}

Turkish uses \textit{-lAr} and \textit{-Iz} suffixes to mark the number information \citep{GokselKerslake2005}. The morpheme \textit{-Iz} only surfaces with first-person and second-person plural while \textit{-lAr} surfaces with the third-person plural. None of the experimental or filler items contain first-person and second-person pronouns in this thesis. Thus, we are only interested in \textit{-lAr}.\footnote{Turkish has two types of plurality marking: additive and associative, both of which are marked with \emph{-lAr}. One way to distinguish between two plurals is to use with possessive marking. While \emph{anne-m-ler} (after the first person possessive) can be translated as my mom and her associates, \emph{anne-ler-im} (before the first person possessive) can be translated as my moms. See \citeand{Furkan2021} for further discussion and why they do not have to be treated separately.}

The verb in Turkish may be marked overtly when the subject is a plural entity. However, this marking is not obligatory in Turkish \citep{GokselKerslake2005}. Both (\ref{ex:OvertPlGrammaticall}) and (\ref{ex:CovertPlGrammatical}) are grammatical since plural marking at the verb is optional in Turkish. 

\ea \label{ex:TurkishPluralOptionality}
  \ea \label{ex:OvertPlGrammaticall}
    \gll \c{C}ocuk-lar okul-a git-ti-ler.\\
    kid-\Pl{}[\Nom{}] school-\Dat{}[\Sg{}] go-\Pst{}-\Tpl{}\\
    \glt `Kids went to school.'
  \ex \label{ex:CovertPlGrammatical}
    \gll \c{C}ocuk-lar okul-a git-ti.\\
    kid-\Pl{}[\Nom{}] school-\Dat{}[\Sg{}] go-\Pst{}[\Tpl{}]\\
    \glt `Kids went to school.'
  \z
\z


This optionality is only relevant when the subject is plural. When the subject is singular, the verb cannot have a plural marking \emph{-lAr} as in (\ref{ex:OvertPlUngrammatical}).

\ea[*]{\label{ex:OvertPlUngrammatical}
  \gll \c{C}ocuk okul-a git-ti-ler.\\
  kid[\Nom{}.\Sg{}] school-\Dat{}[\Sg{}] go-\Pst{}-\Tpl{}\\
  \glt `Kid went\textsubscript{\Pl{}} to school.'}
\z

Moreover, Turkish verbs have to be marked with an overt plural morpheme when the subject is pro-dropped; thus, retrieved from the context and not readily available in the sentence. Consider (\ref{ex:proOblLarGrammatical}) and (\ref{ex:proOblLarUngrammatical}), where the first sentence provides the plural entity \textit{\c{c}ocuklar}. The subject of the second sentence is dropped and represented with \textit{pro\textsubscript{i}}. The coindexation with the subscript \emph{i} represents that the sick ones from the first sentence. 

\ea \label{ex:proObligatoryLar}
  \ea \label{ex:proOblLarGrammatical}
    \gll \c{C}ocuk-lar\textsubscript{i} okul-a git-mi\c{s}(-ler)-di. \emph{pro\textsubscript{i}} Hasta-lan-m{\i}\c{s}-lar.\\
    kid-\Pl{}[\Nom{}] school-\Dat{}[\Sg{}] go-\Evid{}(-\Pl{})-\Pst{} \emph{pro\textsubscript{i}} sick-\Vblz{}-\Evid{}-\Pl{}\\
    \glt `Kids went to school. They got\textsubscript{\Pl{}} sick.'
  \ex[*]{\label{ex:proOblLarUngrammatical}
    \gll \c{C}ocuk-lar\textsubscript{i} okul-a git-mi\c{s}(-ler)-di. \emph{pro\textsubscript{i}} Hasta-lan-m{\i}\c{s}.\\
    kid-\Pl{}[\Nom{}] school-\Dat{}[\Sg{}] go-\Evid{}(-\Pl{})-\Pst{} \emph{pro\textsubscript{i}} sick-\Vblz{}-\Evid{}[\Sg{}]\\
    \glt Intended: `Kids went to school. They got\textsubscript{\Sg{}} sick.'}
  \z
\z


One other aspect of Turkish number agreement is that plurality is not accessible even when the nouns are notionally plural. They cannot be used with phrases like \emph{birbirleriyle} (\emph{each other}), nor with the plural marking at the verb as in (\ref{ex:NotPlEachOther}) and (\ref{ex:NotPlVerbal}) \citep{Sag}. %cite yagmur sag?

\ea \label{ex:notionalPlural}
  \ea[*]{\label{ex:NotPlEachOther}
    \gll Aslan birbirleriyle sava\c{s}-{\i}r.\\
    lion[\Sg{}] each\_other fight-\Aor{}[\Sg{}]\\
    \glt Intended: `Lions fight with each other.'}
  \ex[*]{\label{ex:NotPlVerbal}
    \gll Aslan orman-{\i} koru-r-lar.\\
    lion[\Sg{}] forest-\Acc{} protect-\Aor{}-\Pl{}\\
    \glt Intended: `Lions protect the forest.'}
  \z
\z


However, not every \emph{-lAr} provides plurality meaning. The verbal plural morpheme is also used as the honorific marker (\ref{ex:larPolite}). However, when used as an honorific marker, the sentence includes various other elements that emphasize this formal setting, such as \textit{bey} (\emph{sir}), \textit{efendim} (\emph{sir}) or \textit{han{\i}m} (\emph{Mrs.}).

\ea \label{ex:larPolite}
  \gll Doktor Han{\i}m gel-di-ler efendi-m.\\
  doctor Mrs. come-\Pst{}-\Hon{} sir-\Poss.\Fsg{}\\
  \glt `Mrs. Doctor has arrived, sir.'
\z


\subsection{Possessive constructions}

Another important morpho-syntactic aspect of Turkish for agreement attraction studies is the possessive constructions. Turkish has three different possessive constructions: genitive-possessive constructions (GP), possessive free genitives (PFG), and possessive compounds (PC) as in (\ref{ex:gp}), (\ref{ex:pfg}), and (\ref{ex:pc}), respectively. In this thesis, we only use genitive-possessive constructions. 

\ea \label{ex:possConstructions}
  \ea \label{ex:gp}
    \gll Adam-{\i}n araba-s{\i}\\
    man-\Gen{} car-\Poss{}\\
    \glt `the man's car'
  \ex \label{ex:pfg}
    \gll Adam-{\i}n araba\\
    man-\Gen{} car\\
    \glt `the car of the man'
  \ex \label{ex:pc}
    \gll Adam araba-s{\i}\\
    man car-\Poss{}\\
    \glt `man's car'
  \z
\z

As seen in (\ref{ex:gp}), GP can be seen as a Turkish equivalent of the Saxon Genitive, in which the possessor is marked with the genitive case and the possessee with the possessive marker. Although possessive suffix agrees with the possessor's grammatical person with pronominal forms as in Table \ref{tab:possAgreement}, we are not concerned with any of the allomorphy here since we never utilize pronominal forms in our experiments. 

\begin{table}[hbt!]
  \caption{Genitive-Possessive Agreement Allomorphy}
  \vspace{10pt}
  \begin{tabular}{lrlrl}
    \hline
                            & \multicolumn{2}{l}{Possessor} & \multicolumn{2}{l}{Possessee}   \\ \hline
    \Fsg{} & ben&-{im}    & kitab&-{{\i}m}    \\
    \Ssg{} & sen&-{in}    & kitab&-{{\i}n}    \\
    \Tsg{} & on&-{un}     & kitab&-{{\i}}     \\
    \Fpl{} & b&-{iz-im}   & kitab&-{{\i}m-{\i}z} \\
    \Spl{} & s&-{iz-in}   & kitab&-{{\i}n-{\i}z} \\
    \Tpl{} & on&-{lar-{\i}n} & kitap&-{lar-{\i}} \\ \hline
  \end{tabular}

  \label{tab:possAgreement}
\end{table}

In this thesis, three aspects of possessive constructions will be essential for us: (i) the floating consonant of the possessive (\emph{s}), (ii) the genitive case's subject marking use, and (iii) the specificity of the possessive marked possessee. 

When we contrast the word \emph{kitab-{\i}} from Table \ref{tab:possAgreement} and \emph{araba-s{\i}} from (\ref{ex:gp}), we see that the possessive marking has two distinct forms.\footnote{We are aware that the possessive marking has eight different forms when the vowel harmony facts of Turkish is taken into account. However, for our purposes, we focus on the alternation between the form with an initial consonant and the form without it.} While the form following a consonant-final word (\emph{-I}) is ambiguous between the possessive marking and the accusative marking, the form following a vowel-final word (\emph{-sI}) is not ambiguous. It can only be interpreted as a possessive marking. This is because the floating consonant of the accusative case is \emph{y} and not \emph{s}. 

Considering that the genitive-marking is the default case for specific subjects in embedded clauses, the phrase \emph{onun kitab{\i}} in (\ref{ex:gp}) becomes locally ambiguous. The marking on the noun \emph{kitab} can either be the accusative case (\ref{ex:accParse}) or the possessive marker (\ref{ex:possParse}), but this is unknown until a disambiguating verb phrase is encountered. If the verb phrase is marked with a nominalizer and the argument structure is available, we can have parse as in (\ref{ex:accParse}) where the genitive marked DP \emph{onun} is the subject of the embedded clause, and the word \emph{kitab{\i}} is marked with the accusative case and it is the object of the embedded clause. If the disambiguating verb phrase is a matrix verb, then the genitive marked DP \emph{onun} is the possessor in the genitive-possessive construction, and the word \emph{kitab{\i}} is marked with the possessive marker.


\ea \label{ex:difParses}
  \ea \label{ex:accParse}
    \gll On-un kitab{-{\i}} oku-yaca\u{g}-{\i}n-{\i} d{\"u}\c{s}{\"u}n-m-{\"u}yor-um.\\
    \Tsg{}-\Gen{} book{-\Acc{}} read-\Fut{}-\Poss{}-\Acc{} think-\Neg{}-\Impf{}-\Fsg{}\\
    \glt `I do not think he will read the book.'
  \ex \label{ex:possParse}
    \gll On-un kitab{-{\i}} \c{c}ok ak{\i}c{\i}-y-m{\i}\c{s}.\\
    \Tsg{}-\Gen{} book{-\Poss{}} very smooth-\Cop{}-\Evid{}\\
    \glt `Apparently, her book is really smooth.'
  \z
\z

The last significant aspect of the possessive constructions is their interaction with the differential object marking. Turkish employs differential object marking, and the criterion Turkish speakers use is specificity \citep{Enc1991,HeusingerBamyaci2017,HeusingerKornfilt2005}. When a direct object is a specific noun, it is marked with the overt accusative case. In Turkish GPs, all possessee nouns are specific nouns \citeand{OzturkTaylan2016}. Due to their specificity, when they are direct objects, they have to be marked with the accusative case overtly as in (\ref{ex:possAcc}). Even though Turkish allows bare objects, inherently specific nouns and pronouns must be marked with the accusative case \citep{Kelepir2001}. Similarly, the genitive-possessive constructions cannot be bare when they are in an object position. Thus, whenever we have a bare GP, it has to be the subject of the phrase. 

\ea \label{ex:possAcc}
  \ea \label{ex:possAccGrammatical}
    \gll Mary John-un araba-s{\i}n-{\i} be\u{g}en-di.\\
    Mary John-\Gen{} car-\Poss{}-\Acc{} like-\Pst{}\\
    \glt `Mary liked John's car.'
  \ex[*]{\label{ex:possAccUngrammatical}
    \gll Mary John-un araba-s{\i} be\u{g}en-di.\\
    Mary John-\Gen{} car-\Poss{} like-\Pst{}\\
    \glt Intended: `Mary liked John's car\textsubscript{\it non-specific}.'}
  \z
\z


\subsection{Relative clauses}

The last aspect of Turkish morpho-syntax that will be used in this thesis is the relative clauses. Turkish relative clauses typically precede the head they modify as in (\ref{ex:typicalRC}).\footnote{Some marked constructions with the complementizers \emph{ki} and \emph{hani} can introduce post-nominal relative clauses as well.} The subject of the relative clause is marked with the genitive case when the subject is specific. The subject specificity also affects the nominalizer used in relative clauses. With specific subjects, \emph{-dIK} suffix is used as in (\ref{ex:typicalRCSpec}), whereas \emph{-An} is used with non-specific subjects as in (\ref{ex:typicalRCnoSpec}). Another possible nominalizer in relative clauses is \emph{-AcAK}, which always has a genitive-marked subject (\ref{ex:typicalRCECEK}). In this thesis, we always use relative clauses with \emph{-dIK} nominalizers.  


\ea \label{ex:typicalRC}
  \ea \label{ex:typicalRCSpec}
    \gll H{\i}rs{\i}z-{\i}n gir-di\u{g}-i ev g{\"u}zel-mi\c{s}.\\
    thief-\Gen{} enter-\Nmlz{}-\Poss{} home beautiful-\Evid{}.\\
    \glt `The house that the thief broke into was beautiful.'
  \ex \label{ex:typicalRCnoSpec}
    \gll H{\i}rs{\i}z gir-en ev g{\"u}zel-mi\c{s}.\\
    thief enter-\Nmlz{} home beautiful-\Evid{}.\\
    \glt `The house that a thief broke into was beautiful.'
  \ex \label{ex:typicalRCECEK}
    \gll H{\i}rs{\i}z-{\i}n gir-ece\u{g}-i ev g{\"u}zel-mi\c{s}.\\
    thief enter-\Nmlz{} home beautiful-\Evid{}.\\
    \glt `The house that the thief would break into was apparently beautiful.'
  \z
\z

Another critical aspect of the Turkish relative clauses is that they may consist of only one element: the verb. All the other elements, including the subject, the direct object, and the indirect object, can be dropped as in (\ref{ex:dropRC}), given that the accommodating context is sufficient.

\ea \label{ex:dropRC}
  \ea \label{ex:fullRC}
    \gll Mary-nin okul-dan tan{\i}-d{\i}\u{g}-{\i} \c{c}ocuk \c{s}imdi {\"u}nl{\"u} bir profes{\"o}r ol-mu\c{s}.\\
    Mary-\Gen{} school-\Abl{} know-\Nmlz{}-\Poss{} kid now famous a professor be-\Evid{}\\
    \glt `The kid that Mary used to know from the school is now a famous professor.'
  \ex \label{ex:droppedRC}
    \gll Tan{\i}-d{\i}\u{g}-{\i} \c{c}ocuk \c{s}imdi {\"u}nl{\"u} bir profes{\"o}r ol-mu\c{s}.\\
    know-\Nmlz{}-\Poss{} kid now famous a professor be-\Evid{}\\
    \glt `The kid that (he) used to know is now a famous professor.'
  \z
\z

Lastly, in this thesis, we use object relative clauses as in (\ref{ex:typicalorc}), rather than subject relative clauses as in (\ref{ex:typicalsrc}), both of which are possible in Turkish. 

\ea \label{ex:srcandorc}
  \ea \label{ex:typicalorc}
    \gll H{\i}rs{\i}z-{\i}n \c{c}al-d{\i}\u{g}-{\i} elbise-yi sev-iyor-du-m.\\
    thief-\Gen{} steal-\Nmlz-\Poss{} dress-\Acc{} love-\Impf-\Pst-\Fsg{}\\
    \glt `I used to love the dress which the thief stole.'
  \ex \label{ex:typicalsrc}
    \gll Elbise-yi \c{c}al-an h{\i}rs{\i}z-{\i} tan{\i}-yor-du-m.\\
    dress-\Acc{} steal-\Nmlz{} thief-\Acc{} know-\Impf-\Pst-\Fsg{}\\
    \glt `I used to know the thief who stole the dress.'
  \z
\z

\section{Overview}

This thesis is organized as follows. We begin with a summary of the agreement attraction accounts in Chapter \ref{ch:accounts}. The same chapter introduces several essential topics such as case syncretism, form heuristics, shallow processing, and response bias. In Chapters \ref{ch:exp1}, \ref{ch:exp2}, \ref{ch:exp3}, and Appendix \ref{ap:exp4}, we report our speeded-acceptability judgment experiments on the previously introduced topics, respectively. We summarize and visualize our results in these chapters, and discuss how we interpret our results. Chapter \ref{ch:exp3} also provides details and justifications of our proposed bias calculation. We contextualize and discuss our results in Chapter \ref{ch:discussion} and provide a conclusion in Chapter \ref{ch:conclusion}. We also provided an additional experiment and its analysis in Appendix \ref{ap:exp4} where we controlled for the role of register in agreement attraction.

\section{Data availability}

The data for this thesis, along with our analysis scripts can be found at https://github.com/utkuturk/ma-thesis.