









\chapter{CONCLUSION} \label{ch:conclusion}

This thesis aims to contribute to the broader question in psycholinguistics: What is the role of non-linguistic components, such as form ambiguity, task effects, and response bias, in language comprehension? We use the agreement attraction phenomenon in Turkish to test these components.

Our experimental results, along with the previous work, provided evidence that simple heuristics based on the form are quickly overwritten by syntactic information.

Ambiguous case marking on the head noun does not seem to affect agreement attraction in Turkish. However, \citet{Slioussar2018} showed that even the immediately resolved form ambiguity on the attractor increased attraction effects. The form ambiguity was effective when it was introduced on the attractor but not on the head, a syntactically more prominent position. 

Moreover, form-related strategies have impacted experimental results in various phenomena. Nevertheless, verbal attractors in Turkish did not function as attractors, unlike nominal attractors, even though both are possible agreement controllers in Turkish. 

However, we also showed that the response bias, another non-linguistic component, is the main reason behind the so-called grammaticality asymmetry. This asymmetry previously supported a trend in sentence processing that utilizes content-addressable memory architecture and a cue-based retrieval system. With our findings coupled with \cites{HammerlyEtAl2019} results, we established that the grammaticality asymmetry is not a direct consequence of sentence processing mechanisms but a complication due to the experimental process.

We conclude that the syntactically more informed theory of attraction, the Marking and Morphing account, explains our results better and accounts for the interaction between agreement attraction and non-linguistic components in language comprehension.


% endmatter
